% !TEX encoding = IsoLatin

% La riga soprastante serve per configurare gli editor TeXShop, TeXWorks
% e TeXstudio per gestire questo file con la codifica IsoLatin o Latin 1
% o ISO 8859-1.

% per commentare una riga mettere % al suo inizio
% per s-commentare una riga (ossia attivarla) togliere il % al suo inizio
%
\documentclass[pdfa% formato PDF/A, obbligatorio per l'archiviazione delle tesi di Polito
,cucitura%lascia margine per la rilegatura
%,twoside% per stampa fronte-retro (fortemente consigliato per tesi voluminose, opzionale per le altre)
%,12pt% font pi� grande (12pt) rispetto a quello normalmente usato (11pt)
]{toptesi}
%
\usepackage[latin1]{inputenc}% IMPORTANTE! usare codifica ISO-8859-1 per le lettere accentate
\usepackage{booktabs}

\usepackage{array}
\usepackage{longtable}
\usepackage{booktabs}
\usepackage{enumitem} 
\usepackage{ltxtable}

\usepackage{hyperref}
\hypersetup{%
    pdfpagemode={UseOutlines},
    bookmarksopen,
    pdfstartview={FitH},
    colorlinks,
    linkcolor={blue},
    citecolor={red},
    urlcolor={blue}
  }
% \documentclass[11pt,twoside,oldstyle,autoretitolo,classica,greek]{toptesi}
% \usepackage[or]{teubner}
%%%%%%%%%%%%%%%%%%%%%%%%%%%%%%%%%%%%%%%%%%%%%%%%%%%%
%
% Esempio di composizione di tesi di laurea.
%
% Questo esempio e' stato preparato inizialmente 13-marzo-1989
% e poi e' stato modificato via via che TOPtesi andava
% arricchendosi di altre possibilita'.
%
% Nel seguito laurea "quinquennale" sta anche per "specialistica" o "magistrale"

% Cambiare encoding a piacere; oppure non caricare nessun encoding se si usano
% solo caratteri a 7 bit (ASCII) nei file d'entrata.
%

\usepackage[
backend = bibtex,
giveninits = true,
style = polito,
maxbibnames = 99,
dateabbrev = false
]{biblatex}

\addbibresource{Bibliografia/biblio.bib}
\input{commands.tex}

\begin{document}

\ateneo{Politecnico di Torino}

%%% scegliere la propria facolt� (solo PRIMA dell'AA 2012-2013)
%
%\facolta[III]{Ingegneria dell'Informazione}
%\facolta[IV]{Organizzazione d'Impresa\\e Ingegneria Gestionale}
%\Materia{Remote sensing}% uso sconsigliato

%\monografia{Gestione informatizzata di un magazzino ricambi}% per la laurea triennale
\titolo{Sviluppo di un Serious Game in ambito di Sicurezza Informatica}% per la laurea quinquennale e il dottorato
%\sottotitolo{Metodo dei satelliti medicei}% NON obbligatorio, per la laurea quinquennale e il dottorato

%%% scegliere il proprio corso
%
%\corsodilaurea{Ingegneria dell'Organizzazione d'Impresa}% per la laurea di primo e secondo livello
%\corsodilaurea{Ingegneria Logistica e della Produzione}% per la laurea di primo e secondo livello
%\corsodilaurea{Ingegneria Gestionale}% per la laurea di primo e secondo livello
\corsodilaurea{Ingegneria Informatica}% per la laurea di primo e secondo livello
%\corsodidottorato{Meccanica}% per il dottorato

\candidato{Andrea \textsc{Marchetti}}% per tutti i percorsi
%\secondocandidato{Evangelista \textsc{Torricelli}}% per la laurea magistrale solamente
%\direttore{prof. Albert Einstein}% per il dottorato
%\coordinatore{prof. Albert Einstein}% per il dottorato
\relatore{prof.\ Antonio Lioy}% per la laurea e il dottorato
\secondorelatore{ing.\ Andrea Atzeni}% per la laurea magistrale
%\terzorelatore{{\tabular{@{}l}dott.\ Neil Armstrong\\prof. Maria Rossi\endtabular}}% per la laurea magistrale
%\tutore{ing.~Karl Von Braun}% per il dottorato
%\tutoreaziendale{dott.\ ing.\ Giovanni Giacosa} % solo per la laurea di secondo livello con tesi svolta in azienda
%\NomeTutoreAziendale{Supervisore aziendale\\Centro Ricerche FIAT}
%\sedutadilaurea{Agosto 1615}% per la laurea quinquennale
%\esamedidottorato{Novembre 1610}% per il dottorato
%\sedutadilaurea{\textsc{Novembre} 2017}% per la laurea triennale
\sedutadilaurea{\textsc{Anno~accademico} 2018-2019}% per la laurea magistrale
%\annoaccademico{1615-1616}% solo con l'opzione classica
%\annoaccademico{2006-2007}% idem
%\ciclodidottorato{XV}% per il dottorato
\logosede{Immagini/logopolito}
%
%\chapterbib %solo per vedere che cosa succede; e' preferibile comporre una sola bibliografia
%\AdvisorName{Supervisors}
%\newtheorem{osservazione}{Osservazione}% Standard LaTeX

%\usepackage[a-1b]{pdfx}
%\hypersetup{%
%    pdfpagemode={UseOutlines},
%    bookmarksopen,
%    pdfstartview={FitH},
%    colorlinks,
%    linkcolor={blue},
%    citecolor={green},
%    urlcolor={blue}
%  }

%
% per numerare e far comparire nell'indice anche le sezioni di quarto livello
% SCONSIGLIATO! da usarsi solo in caso di estrema necessit�
%\setcounter{secnumdepth}{4}% section-numbering-depth
%\setcounter{tocdepth}{4}% TOC-numbering-depth (TOC=Table-Of-Content)

%\setbindingcorrection{3mm}

\errorcontextlines=9

\frontespizio
\paginavuota
\newpage
%per sfruttare meglio lo spazio nella pagina
\advance\voffset -5mm
\advance\textheight 30mm

% opzionale, solo se si vuole dedicare la tesi a delle persone care
%\begin{dedica}
%A mio padre

%\textdagger\ A mio nonno Pino
%\end{dedica}

\sommario

%\ringraziamenti

%% inserire sempre nella tesi per la laurea di I livello, perch� il nome dei tutori non � indicato sul frontespizio.
%Il lavoro descritto in questa monografia � stato svolto sotto la supervisione
%del Prof. Antonio Lioy (tutore accademico)% inserire sempre il nome del tutore accademico
% e dell'Ing. Mario Rossi (tutore aziendale)% inserire solo se la monografia � relativa ad un tirocinio.
%.

%\tablespagetrue % normalmente questa riga non serve ed e' commentata
%\figurespagetrue % normalmente questa riga non serve ed e' commentata

\indici

\mainmatter

\chapter{Introduzione}

\input{Capitoli/introduzione.tex}

\chapter{Serious Games}

\input{Capitoli/serious_games.tex}

\chapter{Sistemi SCADA}

Con questo capitolo si cercher� di fornire un'analisi generale sui sistemi SCADA. Innanzitutto, nella sezione \ref{sec:whatis} verr� data la definizione generale di cosa sia un sistema SCADA e per quali scopi � principalmente utilizzato. Nella sezione \ref{sec:architecture} verr� fornita una descrizione dell'architettura generica di un sistema di questo tipo, analizzando le componenti presenti in un sistema SCADA tipo e come si affronta la loro implementazione. Infine, nella sezione \ref{sec:scadaevolution}, si confronter� gli SCADA con una tipologia di sistemi di controllo simili, i DCS. Verr� fornita anche una panoramica sull'evoluzione tecnologica che ha inevitabilmente influenzato anche il campo dei sistemi di controllo e come questa abbia portato alla definizione del concetto di ``servizi SCADA''.

\section{Cosa � un sistema SCADA\label{sec:whatis}}

La definizione di sistema SCADA � contenuta all'interno dello stesso acronimo che viene utilizzato per identificare un tipo particolare di sistemi di controllo industriale, o ICS (Industrial Control System).  SCADA � l'abbreviazione di ``Supervisory Control and Data Acquisition'', ovverosia un sistema il cui scopo � di supervisione, controllo e acquisizione dei dati (funzionale ai fini dello svolgimento delle altre due operazioni). Gli scenari di utilizzo sono i pi� vari: dai classici sistemi di distribuzione e/o produzione energetica (impianti nucleari), ai sistemi di controllo fluidi (impianti di gestione della rete idrica/fognaria), ai sistemi di controllo del traffico, ferroviario e/o automobilistico, fino a sistemi geograficamente pi� contenuti, ma che necessitano comunque di controlli costanti (impianti di produzione industriale, stazioni di servizio, ecc.). 

Dalla definizione data, per�, non si riesce ad estrapolare cosa differenzi un sistema SCADA da un generico impianto di controllo distribuito (Distributed Control System, da qui in avanti DCS). Esistono infatti numero sistemi di controllo la cui classificazione � basata su diversi parametri, quali complessit� dei processi controllati, distribuzione geografica pi� o meno ampia, distribuzione dell'intelligenza di controllo, tempo di reazione disponibile al verificarsi di un evento prodotto dal sistema stesso, modalit� di interazione tra umani e macchinari e molti altri fattori ancora. Per capire veramente queste differenze occorre innanzitutto analizzare nel dettaglio come vengono svolti i tre compiti fondamentali (acquisizione dati, supervisione e controllo) e quali elementi, in generale, sono presenti nell'architettura di un sistema SCADA.

\subsection{Supervisione (Supervisory)}

La supervisione � la funzione principale a cui ogni sistema SCADA deve asserire (� possibile affermare che, senza supervisione, un impianto di controllo non pu� essere classificato come sistema SCADA). Tramite questa funzione � possibile monitorare lo stato in cui si trova il processo controllato e quale sar� la sua evoluzione. Per tale scopo, sono implementate tutte quelle funzionalit� che permettono di visualizzare le informazioni che descrivono lo stato attuale del processo, cos� come sono utilizzate delle strutture dati al cui interno sono raccolte le cosiddette informazioni storiche, che descrivono tutti i possibili stati assumibili dal processo durante la sua evoluzione. Tutto ci� � particolarmente utile nel momento in cui occorre identificare un eventuale stato anomalo del processo.

\subsection{Controllo (Control)}

Il controllo � la funzione che permette al sistema SCADA di prendere delle decisioni, in relazione allo stato attuale del processo controllato e alle sue future evoluzioni. Ovviamente questo compito pu� essere svolto in modalit� differenti, ma la cui definizione dipende principalmente dal tipo di processo controllato, a seconda del quale � necessario creare un'architettura sia hardware che software specifica per il compito. Una concetto molto importante � che in un sistema SCADA il controllo del processo � concentrato per la maggior parte all'interno dell'unit� di elaborazione (controllo centralizzato). In questo senso, l'elaboratore si serve del sistema di acquisizione dati per ottenere le informazioni ottenute dalla funzione di supervisione. Una volta elaborati queste dati grezzi (che non sono nient'altro che una rappresentazione dello stato del processo) se � necessario, cambia il valore dei parametri di stato che definiscono il processo di controllato, sfruttando sempre il sistema di acquisizione, ma nel senso opposto.

\subsection {Acquisizione Dati (Data Acquisition)}

L'acquisizione dei dati, sebbene sia di supporto alle precedenti funzioni, � quella che ricopre all'interno di un sistema SCADA una posizione fondamentale. � importante sottolineare che acquisizione dati, in questo contesto, sta ad identificare uno scambio bidirezionale tra l'unit� di controllo e quella di supervisione. Vi sono casi in cui l'acquisizione dati � il compito principale di un sistema SCADA, poich� il controllo e la supervisione possono essere realizzate in maniera pi� superficiale o anche a posteriori. Esempi di questi tipo sono forniti sono i sistemi di telerilevamento, in cui si raccolgono dati che verranno, eventualmente, con modalit� non sempre costanti.

In generale, per�, l'acquisizione ha un ruolo funzionale allo svolgimento degli altri due compiti di un sistema SCADA. Senza scambio dei dati, infatti, l'unit� di elaborazione non pu� avere le informazioni sullo stato attuale del processo, cos� come il sistema di supervisione non ricever� i valori dei parametri di controllo con cui gestire l'evoluzione dello stato del processo. 

Lo scambio deve avvenire nella modalit� pi� semplice possibile, senza alcuna manipolazione e/o elaborazione di sorta durante il percorso. Il processo di controllo avr� luogo all'interno del sistema di elaborazione. Ovviamente esistono situazioni in cui, le elevate dimensioni geografiche del processo controllato non permettono l'utilizzo di una sistema di controllo centralizzato, ma richiedono l'utilizzo di un sistema ad ``intelligenza distribuita''. In questo caso, per�, non siamo pi� di fronte ad un sistema SCADA.

\subsection{Analisi del processo da controllare}

L'architettura di un sistema SCADA generalmente � sempre la stessa ed � costituita da delle componenti che si possono definire ``standard''. Al momento della definizione delle caratteristiche funzionali di un sistema SCADA, per�, non si pu� ignorare il fattore rappresentato dal processo controllato. A seconda del tipo di analisi che si vuole condurre su di esso, infatti, si vanno a definire i vari parametri progettuali dell'impianto SCADA, sia quelli di tipo tecnologico che di tipo organizzativo. In questo senso, � utile classificare il processo da controllare in base a delle caratteristiche che aiuteranno poi a definire le funzioni fondamentali del sistema di controllo. \pagebreak

\begin{description}

\item [Tempo di reazione]
 Con questo parametro si stabilisce con quanto ritardo � in grado di reagire il sistema di controllo ai cambiamenti di stato del processo durante la sua evoluzione. Solitamente, nei sistemi SCADA, si parla di reazioni ``real-time'', ossia che abbiano un ritardo trascurabile, anche se pu� capitare che questa caratteristica vada in contrasto con altri aspetti legati al sistema di controllo. Il primo tra questi � il limite dovuto alla tecnologia con cui si realizza il sistema di acquisizione dati. Un esempio di quanto detto � dato dagli impianti di controllo con grandi dimensioni geografiche dove, a causa di ci�, il trasporto delle informazioni avr� un tempo che non sar� mai non trascurabile ed andr� ad introdurre un ritardo nelle reazioni dell'intero sistema SCADA. Un'altra criticit� pu� essere data dalla complessit� dell'unit� di elaborazione, spesso necessaria per soddisfare i vincoli di affidabilit� e disponibilit�, necessari per una corretta operativit� dell'intero sistema SCADA.
 Per cercare di conciliare tutti i vari requisiti sono adottate diverse soluzioni, a seconda del risultato pi� conveniente per il caso in cui operer� il sistema SCADA:
 
 \begin{itemize}
 
 \item ridurre i tempi di reazione solo per una certa parte degli eventi generati da un processo;
 
 \item ridurre la complessit� delle funzioni implementate nell'unit� di elaborazione, e pi� in generale la complessit� dell'intero sistema SCADA;
 
 \item utilizzare soluzioni tecnologiche ad hoc per il contesto in cui il sistema SCADA operer�.
 
 \end{itemize}
 
\item [Affidabilit�]
L'affidabilit� (intesa come ``reliability'') � un parametro altrettanto importante quando si parla di sistemi SCADA, di cui non si pu� fare a meno. Essendo utilizzati una grande quantit� di componenti, ognuno dei quali con il proprio grado di affidabilit�, al momento dell'implementazione del sistema occorre tenere conto di ognuno di questi valori. Ci� serve a stabilire le contromisure necessarie per evitare che, nel momento in cui si verifichi un malfunzionamento, questo influenzi l'intero impianto SCADA. Questo discorso vale sia per componenti di terze parti, di cui � sempre consigliabile verificare l'affidabilit� dichiarata dal produttore, sia per parti auto-prodotte, per la cui realizzazione ci si dovrebbe affidare a strumenti adeguati cercando di ottenere un grado di affidabilit� soddisfacente per l'utilizzo a cui � destinato.

Una considerazione da fare, quando si parla dell'affidabilit�, � che, in alcuni casi, cercare di ottenere un grado elevato di questa propriet� pu� risultare troppo oneroso e con risultati non sempre all'altezza della cifra investita. Un eventuale guasto, infatti, potrebbe anche non pregiudicare il corretto funzionamento di un sistema di controllo, se questo avviene entro particolari condizioni. Un esempio � dato dagli impianti di rilevamento ambientale, in cui in caso di perdita di dati, si pu� parlare di evento trascurabile, se l'intervallo di tempo interessato dal disservizio � breve rispetto al tempo totale in cui il sistema ha lavorato.
 
\item [Disponibilit�]
Con questo parametro si indica il tempo totale in cui il sistema di controllo ha garantito il corretto funzionamento del processo controllato. Ovviamente, come nel caso dell'affidabilit�, anche la disponibilit� si pu� riferire all'intero sistema come alle singoli parti che lo compongono. In generale il soddisfacimento dei vincoli imposti dipende dal tipo di processo controllato: si possono avere situazioni con vincolo molto stringenti, cos� come processi per cui la disponibilit� � una caratteristica che passa in secondo piano (ma comunque si avranno sempre dei vincoli da rispettare al momento dell'implementazione del sistema SCADA) 

\item [Interazione uomo-macchina]
Data la presenza di sistemi di controllo e supervisione, � necessario implementare anche un sistema con cui si fornisce ad un operatore la possibilit� di interfacciarsi con l'impianto SCADA. Questi sono spesso indicate con la sigla HMI (Human Machine Interface), la cui complessit� dipende, come per tutti gli altri parametri, dal processo controllato dal sistema SCADA. Per impianti in cui sono utilizzate procedure completamente automatizzate, baster� un'interfaccia che sia semplicemente in grado di permettere l'osservazione dei vari stati evolutivi del processo. D'altra parte, nei casi in cui, oltre all'osservazione, � richiesto all'operatore di effettuare anche operazioni di controllo, � necessario un'interfaccia che implementi tutte le funzionalit� richieste per svolgere il compito assegnato, senza per� andare ad inficiare sulla facilit� di utilizzo, che deve rimanere in ogni caso adeguatamente alta.

In passato, vi erano casi particolari in cui era impossibile realizzare un qualsiasi tipo di interfaccia, per via delle numerose informazioni da riportare a schermo. Oggi, grazie all'utilizzo di monitor dalle grandi dimensioni o, se necessario, di ``video wall',' questi casi sono sempre meno frequenti.

\item [Dimensione geografica]
Anche in questo caso, la dimensione del sistema � strettamente vincolata dalla dimensione geografica in cui ha luogo lo svolgimento processo. Se, ad esempio, parliamo di un processo con un'area geografica limitata, come pu� essere una linea di produzione industriale o un sistema di depurazione, il sistema SCADA pu� essere collocato interamente in un unico edifico, solitamente lo stesso che ospita il processo da controllare. Se invece prendiamo in considerazione il controllo di un sistema di trasporto energetico, la cui dimensione � chiaramente vasta, � necessario che il sistema SCADA rispecchi la distribuzione geografica del processo stesso, in modo da rendere lo svolgimento del lavoro pi� semplice. Ovviamente, in questo caso, aumenta la complessit� architetturale dell'impianto di controllo, in quanto si rende necessario affrontare le varie criticit� che sono introdotte dalla presenza delle notevoli dimensioni in cui tutta l'operazione si svolge. Una su tutti � l'affidabilit� dei sistemi di comunicazione, che deve risultare sempre molto alta, poich� un'eventuale corruzione delle informazioni trasportate potrebbe pregiudicare completamente le operazioni di controllo. Altra criticit� pu� essere rappresentata dalla presenza di pi� postazioni di controllo, soluzione utilizzata in particolari casi di processi dalle elevate dimensioni geografiche. In questo caso occorre realizzare un adeguato sistema di interfacce uomo-macchina che permette l'accesso contemporaneo ai dati, senza incorrere in errori dovuti all'inconsistenza delle informazioni salvate nei database. 

\end{description}

\section{Architettura di un sistema SCADA\label{sec:architecture}}

In generale, l'architettura di un sistema SCADA � composta da tre sottosistemi adibiti allo svolgimento dei rispettivi compiti di controllo, supervisione e acquisizione dati. Questi sono denominati come:

\begin{itemize}

\item sistema di elaborazione dati, adibito al controllo del processo;
\item sistema di trasmissione dati, destinato allo svolgimento delle funzionalit� di acquisizione dati;
\item sistema di acquisizione, a cui � assegnato il compito di supervisione.

\end{itemize}

Come � possibile notare in figura \ref{fig:scadaarchitecture}, nella zona pi� periferica � presente il sistema di acquisizione, mentre quella centrale � destinata al sistema di elaborazione. La comunicazione tra questi due sottosistemi � garantita grazie al sistema di trasmissione dati.

\begin{figure}[tbh]

\centerline{\includegraphics[scale=0.8]{Immagini/scada_architecture.pdf}}
\caption{Rappresentazione grafica dell'architettura di un generico sistema SCADA}\label{fig:scadaarchitecture}

\end{figure}

Come per il caso del sistema SCADA in generale, l'implementazione di questi tre sottosistemi � specifica per il tipo di processo a cui sono destinati ( e pu� risultare a volte molto complessa), ma se ne pu� dare comunque una definizione generica, individuando gli elementi che caratterizzano ognuno di essi.

\subsection{Sistema di elaborazione}

\begin{figure}[tbh]

\centerline{\includegraphics[scale=0.75]{Immagini/elaborator_architecture.pdf}}
\caption{Rappresentazione grafica dell'elaboratore di un sistema SCADA}\label{fig:elaboratorarchitecture}

\end{figure}

In \ref{fig:elaboratorarchitecture} � riportato uno schema a blocchi dei componenti presenti in un elaboratore. Come � possibile vedere, in generale i vari componenti di un elaboratore possono essere raggruppati in tre macro-blocchi: quello preposto alla gestione dei dati, quello il cui compito � di garantire la disponibilit� delle informazione e il blocco elaborativo vero e proprio. Di seguito verranno analizzati in maniera pi� dettagliata le funzionalit� che caratterizzano ognuno dei blocchi. \pagebreak

\begin{description}

\item [Gestore dati]
Il compito delle componenti per la gestione dei dati � quello di comunicare con le apparecchiature periferiche (sia per raccogliere i dati necessari all'elaborazione che per inviare i dati elaborati necessari per le azioni di controllo), trattare i dati per renderli interpretabili dal sistema e archiviare le informazioni, sia grezze che gi� elaborate. Tutte queste operazioni costituiscono il cuore di tutte le alte funzionalit� del sistema SCADA.

I dati riguardanti lo stati del processo da controllare sono ricevuti dalle apparecchiature di controllo e sono subito tradotti nel formato di riferimento del sistema interno (ovviamente, nel momento in cui occorre comunicare con l'esterno avviene il procedimento inverso, in cui le informazioni sono tradotte nel formato adeguato affinch� siano utilizzabili per effettuare le varie azioni di controllo). Il flusso di queste informazioni, sia in entrata che in uscita, ha come punto di raccolta il ``database runtime'', che viene chiamato in questo modo proprio per il fatto che opera in tempo reale con la dinamica evolutiva del processo controllato. � implementabile in vari modi (ad esempio, pu� essere costituito da aree di memoria condivisa tra i vari processi di controllo), ma l'importante � che soddisfi tutte le richieste ricevute da parte del sistema di elaborazione in tempi ridotti. Tramite ci� vengono, quindi, garantite le funzionalit� di controllo e supervisione del sistema SCADA. Ovviamente, per motivi di sicurezza e per permettere lo svolgimento di azioni di supporto al controllo del processo, � consigliabile mantenere questi dati e salvarli per un eventuale utilizzo successivo. A tale scopo viene utilizzato, parallelamente al ``database runtime'', un ``database relazionale'', che rappresenta il componente principale del blocco adibito a garantire la disponibilit� dei dati.

\item[Disponibilit� dati]
Come detto in precedenza, al fine di permettere lo svolgimento di azioni di supporto a quelle di controllo, spesso si rivela utile utilizzare un ``database relazionale'' in cui salvare tutti i dati, si quelli ricevuti dall'esterno che quelli elaborati dal sistema stesso. Le azioni di supporto possono essere le pi� svariate, dalle analisi legate ai dati descrittivi degli stati del processo, in modo da poter svolgere funzioni di correzione preventiva, alle gestione e consultazione di dati dal carattere economico. Ovviamente, in questo caso non � richiesto che il database lavori in tempo reale. Ha molta pi� importanza che le informazioni abbiano un'elevata intelligibilit�, vi siano gli strumenti di accesso necessari per garantire la disponibilit� dei dati e infine che il database riesca a gestire facilmente le grandi quantit� di informazioni salvate al suo interno. Tutto ci� permette agli operatori ed ai sistemi esterni di fare accesso ai dati senza coinvolgere direttamente il sistema di controllo e supervisione.

\item [Elaboratore]
Il blocco dell'elaboratore � quello preposto alla manipolazione ed interpretazione dei dati descrittori dello stato evolutivo del processo. Questo blocco, una volta ricevute le informazioni dall'esterno (fornite dal ``database runtime''), effettua un'analisi su di essi ed invia come risposta una serie di comandi per il controllo del processo. Ovviamente questa analisi pu� essere fatta dagli operatori umani, ma spesso la grande quantit� di dati da analizzare, corrispondente di solito a processi da controllare dall'elevata complessit�, rendono necessario l'utilizzo di procedure di supporto. Queste producono, sostanzialmente, una serie di informazioni aggregate insieme ad una sintesi dei vari comandi da poter inviare al processo controllato. Ovviamente vi sono anche altre esigenze che questi processi elaborativi di supporto devono poter soddisfare, tra queste le pi� importanti sono:

\begin{itemize}

\item generare segnalazioni di eventuali anomale presenti nell'evoluzione del sistema;
\item generare rappresentazioni sintetiche dello stato attuale e dell'evoluzione del processo;
\item interpretare i comandi forniti dall'operatore;
\item realizzare procedure di controllo automatiche.

\end{itemize}

Quest'ultimo punto si rende necessario nei sistemi SCADA adibiti al controllo di processi che non sono direttamente controllabili dagli operatori umani. In questo caso, il sistema compie delle azioni predefinite di controllo automatico con una cadenza periodica.

\end{description}

\subsection{Sistema di acquisizione}

Il sistema di acquisizione dati in un impianto SCADA rappresenta lo strumento tramite cui dialogare con l'esterno. Il suo compito principale � fare da traduttore tra il sistema centrale (l'elaboratore) e quello periferico, convertendo le informazioni analogiche, quali temperatura, pressione e tutte le altre grandezze che descrivono lo stato del processo, in informazioni binarie. Al fine di permettere una comunicazione corretta, � necessario, oltre a stabilire un linguaggio di comunicazione unico per tutto il sistema SCADA, anche definire le modalit� di comunicazione e la codifica da applicare alle informazioni scambiate. Ovviamente, le tipologie di sistemi di acquisizione possono essere delle pi� disparate, a seconda delle caratteristiche considerate per definire l'architettura del sistema SCADA. 

Al fine di individuare l'apparato di acquisizione dati pi� indicato per le proprie esigenze, � utile capire ed individuare la tipologia di informazioni che l'impianto dovr� gestire. Questa analisi pu� essere svolta in base ai seguenti criteri:

\begin{description}

\item [Direzione delle informazioni]
Sulla base di questo criterio � possibile effettuare una duplice distinzione. Abbiamo infatti le ``informazioni in ingresso'', ossia i dati che riceve, sia dal sistema centrale che da quello periferico, che le ``informazioni in uscita'', di nuovo, informazioni che possono essere dirette o al sistema centrale o agli apparati esterni. Ovviamente, � chiaro che vi � una relazione di equivalenza tra i vari dati gestiti. Le informazioni in ingresso dal sistema centrale, infatti, sono le stesse che, opportunamente tradotte, sono dirette in uscita verso l'impianto periferico (vale, chiaramente, anche l'equivalenza opposta, ossia informazioni in ingresso dal sistema periferico diventano quelle in uscita verso l'elaboratore centrale);

\item [Caratteristiche elettriche delle informazioni]
Questo criterio viene applicato quando sono prese in analisi le informazioni relative agli apparati periferici, che siano in entrata o in uscita. Affinch� il sistema riesca ad interpretare le informazioni, � necessario tradurre le grandezze fisiche in un segnale elettrico adeguato. Ci� � reso possibile grazie al lavoro svolto dai trasduttori, mentre per l'operazione inversa � richiesto l'utilizzo degli attuatori. Ovviamente � necessario che tutti gli apparati di acquisizione utilizzino lo stesso tipo di rappresentazione elettrica dei dati.

Vi sono molteplici standard industriali per ognuno dei segnali considerati, che siano digitali, analogici, in ingresso o in uscita. In generale, per i dati digitali, sono utilizzati dei segnali elettrici dal voltaggio variabile (il range va dai \SI{24}{\volt} ai \SI{220}{\volt}) a corrente sia continua che alternata, a seconda di ci� che � pi� adatto allo scenario in analisi. Per le informazioni analogiche, invece, i segnali utilizzati sono differenziati sulla base delle grandezze elettriche. Abbiamo infatti:

\begin{itemize}

\item misure in tensione;
\item misure in corrente;
\item misure in resistenza;
\item misure in termo resistenza;
\item misure in termo coppia.

\end{itemize}

Tutte queste informazioni sono rilevanti quando si va ad eseguire l'analisi per il corretto dimensionamento del circuito elettrico, in quanto descrivono le caratteristiche elettriche, come ad esempio l'assorbimento, dei vari trasduttori e/o attuatori.

\item [Qualit� delle informazioni]
In questo caso, il termine ``qualit�'' sta ad indicare quale � la tipologia dell'informazione gestita ed � necessario definirla al fine di garantire il loro corretto trattamento. � possibile eseguire questa distinzione in base a quattro macro-aree:

\begin{itemize}

\item \textbf{informazioni digitali:}
queste informazioni sono rappresentate come un insieme di dati binari. Sono associate alle grandezze fisiche che descrivono lo stato di un processo;
\item \textbf{informazioni analogiche:}
questa tipologia di dati � utilizzata per rappresentare le grandezze fisiche tramite una serie di valori oscillanti all'interno di un certo range. Sono richieste delle opportune conversioni analogico/digitale affinch� siano utilizzabili dall'elaboratore per svolgere le funzioni di supervisione e controllo;
\item \textbf{informazioni impulsive:}
questo tipo di informazione non � interpretabile in tempo reale. Piuttosto se ne richiede la conoscenza in un determinato arco temporale, in modo da ottenere una corretta rappresentazione della grandezza a cui sono associate;
\item \textbf{informazioni complesse:}
questo tipo di informazione � prodotto, usualmente, da dispositivi complessi con cui il sistema SCADA si interfaccia. Un esempio � fornito dai contatori elettrici di ultima generazione, che producono un'insieme di informazione relativamente a tensione, corrente, potenze, dati economici ed altro ancora, il tutto in relazione a diversi periodi temporali, pi� o meno lunghi. In questo caso, piuttosto che analizzare ed acquisire ognuno di quei dati in maniera indipendente come delle informazioni analogiche, � preferibile utilizzare interfacce apposite, in grado di stabilire la comunicazione con il dispositivo tramite protocolli di comunicazione ad hoc. Questo permette al sistema di acquisizione di comunicare con tutti i dispositivi esterni che utilizzano quel particolare protocollo, aumentando di fatto la compatibilit� del sistema stesso. Una volta stabilita la comunicazione, questa avviene in maniera autonoma, permettendo l'acquisizione delle informazioni. Tra i vari protocolli utilizzati, si segnalano il ``ModBus'', il ``ProfiBus'', il ``CanBus'' ed il ``LonWorks'', che risultano essere quelli pi� diffusi.

\end{itemize}

\end{description}

A livello pratico, questa analisi si traduce nei parametri di programmazioni da applicare ai PLC, ``Programmable Logic Controller''. Questi sono dei veri e propri computer componibili, la cui struttura hardware � adattata al processo da controllare. Il loro compito � gestire i segnali digitali ed analogici che transitano nella rete costituita dai sensori, gli attuatori e il sistema di elaborazione centrale. Negli ultimi anni, grazie anche alle progressive migliorie tecnologiche che ne hanno permesso una riduzione delle componenti fisiche e, conseguentemente, dei costi, si � cominciato ad utilizzare i PLC anche in ambiti domestici. Un esempio � la loro applicazione nei quadri elettrici delle abitazioni per gestire automaticamente i vari impianti presenti nelle case: riscaldamento, irrigazione, rete internet, ecc.

\subsection{Sistema di trasmissione dati}

I componenti sopra elencati necessitano di interfacce adeguatamente implementate per comunicare correttamente tra di loro. In un sistema SCADA occorre garantire la comunicazione tra:

\begin{itemize}

\item sistema di elaborazione e sistema di acquisizione dati;
\item sistema di elaborazione e sistema di gestione dati;
\item processo controllato e dispositivi di interazione (attuatori);
\item dispositivi di interazione e sistema di acquisizione dati.

\end{itemize}

Pu�, inoltre, rivelarsi utile implementare delle interfacce per comunicare con altri elementi esterni, quali sistemi gestionali dell'azienda o sistemi informativi in generale. Ognuna di queste interfacce dovr� essere implementata secondo le caratteristiche pi� adeguate per lo scopo a cui � destinata. Il rischio � quello di compromettere le normali funzionalit� del sistema SCADA o, addirittura, la sua realizzabilit�. Inoltre, in fase di progettazione, � consigliabile tenere conto anche dei possibili sviluppi futuri a cui potrebbe essere soggetto il sistema SCADA. Di seguito sono riportate le caratteristiche su cui basare l'analisi dei protocolli da applicare alle interfacce comunicative che si vogliono implementare.

\begin{description}

\item [Velocit�]
Uno degli aspetti pi� cruciali dei canali di comunicazione � quello di garantire una velocit� sufficientemente elevata, tale da rendere possibile che l'azione di controllo del processo avvenga in tempi ridotti ed adeguati. I vincoli imposti, in questo senso, sono spesso molto restringenti e ci� pu� costringere all'impiego di sistemi periferici ad intelligenza distribuita per compiere le azioni di controllo. Alcuni dei casi in cui questi problemi si verificano con costanza sono i sistemi SCADA con notevoli dimensioni geografiche. In questo caso occorre utilizzare le infrastrutture di comunicazione dei gestori telefonici, che normalmente sono destinate ad un impiego ``general purpose''. Ci� pu� rendere il servizio non sufficientemente adeguato per lo scopo finale e occorre, quindi, ricorrere a sistemi periferici con intelligenza di controllo distribuita, in modo da svolgere le azioni nei tempi richiesti.

Nel caso della comunicazione tra il sistema e le HMI, invece, la comunicazione deve avvenire sempre in ``real-time'', in modo da rendere pi� efficace il lavoro degli operatorio. Questo deve avvenire sia per la visualizzazione dei parametri descrittivi dello stato del processo, che per la risposta alle azioni di controllo attuate dagli operatori.

Infine, la comunicazione con i sistemi esterni � vincolata dal tipo di sistema con cui occorre interfacciarsi. Se � anch'esso un sistema di controllo, allora i vincoli sono gli stessi visti nel caso di comunicazione con il processo, con annesse limitazioni. Se, d'altra parte, si vuole comunicare con un sistema non di controllo, l'interfaccia da implementare non presenta delle particolari restrizioni da rispettare, anzi possono essere considerate come trascurabili.
� importante sottolineare come, di solito, non sia possibile soddisfare tutti i requisiti imposti dal tipo di comunicazione che si vuole implementare, scendendo di fatto ad un compromesso tra protocolli implementati e tecnologia utilizzata.

\item [Sicurezza]
La sicurezza � un aspetto altrettanto importante da considerare al pari della velocit�, soprattutto nel caso in cui le probabilit� di intrusione da parte di soggetti indesiderati siano abbastanza alte. Chiaramente, nel caso di un sistema chiuso, i tentativi di intrusione a cui si � soggetti diventano esigui, ma non bisogna scordare che � sempre presente la possibilit� di un errore umano da parte degli operatori. Si rende necessario, quindi, ricorrere ai ripari preventivamente, cercando di evitare il presentarsi di queste spiacevoli situazioni, che siano causati da attacchi intenzionali o errori in buona fede. 

La gestione della sicurezza deve riguardare sia le comunicazione tra elaboratore e sistema periferico di acquisizione dati, sia tra un sistema SCADA ed un altro, in quanto i entrambi i casi l'alterazione delle informazioni trasportate pu� provocare un comportamento anomalo da parte dell'impianto di controllo.

Tra le soluzioni adottabili, la pi� gettonata � la separazione delle diverse aree di lavoro accessibili al sistema ed agli operatori. Questa separazione pu� avvenire sia fisicamente, cos� come dal punto di vista dell'implementazione logica, ed in generale � dipendete dal tipo di tecnologia implementata per l'interfaccia di comunicazione. 

In ogni caso, il passo pi� importante da compiere � quello di definire un'adeguata politica di sicurezza fin dalle fasi progettuali, cosa che negli ultimi anni � diventata imprescindibile, ma che nel passato era trattato con molta superficialit�. Questo perch� mentre prima i sistemi SCADA erano isolati completamente dal mondo esterno, oggi fanno largo impiego di tecnologie comunicative pubbliche, dal basso costo, ma dalla sicurezza pi� debole. Si � reso necessario, quindi, un cambio di approccio al tema della sicurezza.

\item [Intelligibilit�]
L'intelligibilit� � un parametro importante nel momento in cui si vuole realizzare un sistema che interagisca costantemente con apparecchiature esterne per la supervisione ed il controllo. In questo caso � possibile applicare diverse soluzioni, ma le pi� adatte sono quelle che fanno riferimento a degli standard comunicativi predefiniti. Ci� � facilmente comprensibile sia dal punto di vista tecnologico, funzionale ed economico. 

Tecnologico e funzionale perch� un protocollo proprietario (ossia non standard) non � detto che sia utilizzato dai dispositivi con cui si vuole comunicare, restringendo quindi la gamma di apparecchi utilizzabili. Inoltre, non � detto che tra quelli effettivamente utilizzabili vi sia il dispositivo adatto a soddisfare tutte le esigenze di comunicazione richieste. Infine, economicamente parlando, un protocollo standardizzato ha dietro di s� il supporto di un'intera comunit� scientifica, mentre nel caso di uno proprietario si � costretti a sottostare alle esigenze di mercato dell'azienda padrona del protocollo.

\item [Affidabilit�]
In generale � richiesto che i dati trasmessi all'elaboratore del sistema SCADA mantengano un alto gradi di integrit�. Ci� per rendere possibile una corretta valutazione dello stato evolutivo del processo. Una soluzione potrebbe essere quella di applicare meccanismi di validazione dati nei dispositivi periferici, ma ci� rallenterebbe la risposta generale del sistema all'evoluzione del processo. L'unica opzione rimanente � integrare questi meccanismi nel canale di comunicazione, in modo da riuscire a garantire l'integrit� delle informazioni trasportate.

La soluzione ideale per rispondere a questo problema prevede l'implementazione di tre diverse funzionalit�:

\begin{itemize}

\item rilevazione degli errori;
\item richiesta di ritrasmissione in caso di errori rilevati;
\item ordinamento delle informazione all'interno del flusso dati.

\end{itemize}

Questi meccanismi sono implementabili nelle maniere pi� differenti, facendo ricorso agli algoritmi che pi� si ritengono adatti. Purtroppo, per�, l'introduzione di queste funzioni va ad intaccare la velocit� di comunicazione, richiedendo, quindi, il raggiungimento di un compromesso in grado di garantire un adeguato rapporto tra la rapidit� di scambio delle informazioni e il mantenimento della loro integrit�. Inoltre, per ottimizzare l'efficacia della soluzione adottata, � consigliabile lo svolgimento di un'attenta analisi del sistema di comunicazione in questione. � ovvio infatti che nel caso in cui la comunicazione avvenga su di un mezzo di per s� gi� affidabile, come la fibra ottica, si preferisce evitare l'implementazione di tecnologie per gestire gli errori di trasmissione, in quanto sarebbe praticamente inutilizzato. Se, invece, la comunicazione avviene su di un mezzo meno affidabile, come dei trasmettitori wireless, � fortemente consigliato l'impiego di funzionalit� di rilevamento e correzione degli errori.

Infine, nel caso in cui la comunicazione da stabilire sia tra pi� sistemi SCADA, l'aspetto che si prende pi� in considerazione � l'ottimizzazione dei costi di trasmissione, in quanto, come detto, sono presenti molti meno vincoli sulla velocit� e sull'affidabilit� dei dati scambiati.

\item [Disponibilit�]
In stretta correlazione con l'aspetto dell'affidabilit� vi � quello della disponibilit� dei dati trasmessi. In casi di disservizio del sistema di comunicazione, infatti, anche le operazioni di controllo, direttamente collegate alle comunicazioni, rischiano di subire dei malfunzionamenti. Occorre, quindi, garantire la continuit� della disponibilit� delle informazioni in quanti necessaria per svolgere correttamente le attivit� di controllo (per le quali � fondamentale conoscere in tempo reale lo stato del processo da controllare o, nel caso di attuazione di una politica di controllo, occorre prontamente informare l'attuatore dell'azione che dovr� intraprendere).

Le soluzioni pi� adatte partono tutte dall'assunzione di un principio molto importante: anche un sistema ad alta disponibilit� pu� interrompere il suo normale servizio per un guasto. In questi casi, per prevenire e combattere il verificarsi di una situazione del genere, si richiede l'utilizzo di protocolli comunicativi opportunamente scelti e la realizzazione di sistemi ridonanti, in cui i dispositivi restano a riposo fino al verificarsi di un guasto. In quel momento saranno prontamente messi in azione per sostituire le componenti ordinarie fuori servizio. Questa soluzione � adottata per tutti i dispositivi presenti nel canale di comunicazione.

\item [Supporto dei servizi]
Un'ulteriore aspetto da considerare nell'analisi del sistema di comunicazione � la tipologia delle informazione scambiate. � stato dimostrato, infatti, che a parit� di tipo di dati trasmesso, diverse tecnologie e protocolli offrono prestazioni diverse da loro. Occorre, quindi, analizzare se l'interfaccia scelta � adeguata a garantire una qualit� di comunicazione sufficientemente elevata per il tipo di dato che si vuole trasmettere.

\end{description}

\section{Differenze tra sistemi DCS e sistemi SCADA\label{sec:scadaevolution}}

Le componenti appena analizzate non sono una prerogativa dei soli sistemi SCADA, ma sono utilizzati nei vari sistemi di controllo, tra cui i DCS. Ci� che li differenzia, quindi, non � quali strumenti sono utilizzati, ma piuttosto come vengono implementati. In particolar modo ka disinzione � sul grado di distribuzione dell'intelligenza di controllo. I DCS, acronimo che sta ad indicare i ``Ditributed Control System'', come suggerisce gi� il nome stesso, sono basati sull'impiego di strutture di acquisizione dati, ma con anche un'elevata capacit� elaborative, creando di fatto un paradigma tecnologico contiguo delle funzioni di controllo e acquisizione, Negli SCADA, invece, come abbiamo visto, le due funzioni sono ben distinte ed affidate ad impianti separati, fisicamente e tecnologicamente.

Nel caso DCS, quindi, non si parla di apparecchiature di acquisizione, ma di unit� di elaborazione periferiche, in grado non solo di ricevere dati, ma anche di interpretarli, analizzarli per individuare lo stato attuale in cui il processo si trova e, se necessario, eseguire delle azioni di controllo sul processo. LA complessit� architetturale di queste unit� periferiche, cos� come le funzioni che sono in grado di svolgere, � basata sul tipo di processo che si deve controllare. L'unit� centrale di elaborazione, nel contesto di un sistema DCS, ha quindi il compito di acquisire sia informazioni grezze che informazioni gi� elaborate che danno indicazioni sullo stato delle strutture di controllo.

Osservando, per�, l'evoluzione tecnologica avvenuta negli ultimi anni che ha interessato anche l'ambiente dei sistemi di controllo, � possibile fare un riflessione sulla necessit� di mantenere o meno questa distinzione tra sistemi DCS e SCADA. Inizialmente, la distinzione era dettata dalle differenti caratteristiche tecnologiche che venivano implementate, spesso frutto di scelte obbligatorie per risolvere un particolare problema di controllo. Con lo sviluppo delle infrastrutture di comunicazione e delle tecnologie computazionali la scelta tra un sistema di elaborazione centralizzato con periferiche di acquisizione pure e un sistema con controllo ed elaborazione distribuito anche nelle apparecchiature di acquisizione � diventata sempre pi� una questione legata a fattori quali scalabilit� dell'impianto, grado di manutenzione da garantire e altre caratteristiche non inerenti alla effettiva realizzabilit� del sistema. A tal proposito, si prevede che a causa del continuo progresso tecnologico, la distinzione tra sistemi DCS e SCADA si far� sempre pi� assottigliata fino a sfociare nella creazione di un'unica categoria di classificazione, di cui entrambi faranno parte.

\subsection{Evoluzione tecnologica dei sistemi SCADA: integrazione con sistemi informativi aziendali}
L'evoluzione tecnologica che ha colpito e continua ad interessare i sistemi SCADA ha portato a due principali conseguenze:la prima � stata, come detto, la diminuzione del divario tra la categoria SCADA e quella dei DCS, creando dei sistemi ibridi in cui sono integrate funzionalit� di entrambe le parti. La seconda, altrettanto importante, � stata la definizione di nuove architetture con lo scopo di creare non pi� una paradigma che descriva un sistema fisico, ma piuttosto una funzionalit�. In tal senso, � utile vedere le diverse linee evolutive che sono state applicate per realizzare questa idea.

La prima � stata quella che permettesse l'integrazione tra un sistema SCADA con uno gestionale concorrente. In questo modo � possibile interagire direttamente con un sistema informatico aziendale, con il fine di fornire supporto automatizzando la gestione interna, sia organizzativa che contabile. In figura \ref{fig:integratedsystem} � possibile vedere un esempio di questo caso. Come � possibile osservare,le varie modalit� di comunicazione del sistema SCADA convergono tutte nel sistema informatico aziendale.

\begin{figure}[tbh]

\centerline{\includegraphics[scale=0.75]{Immagini/architettura_sistema_integrato.pdf}}
\caption{Architettura di un sistema SCADA integrato con un sistema aziendale}\label{fig:integratedsystem}

\end{figure}

Una volta progettata l'architettura, con le entit� che ne faranno parte, e le comunicazioni tra di esse, occorre analizzare di quali informazione il sistema aziendale ha bisogno per svolgere correttamente le sue funzioni. Il supporto che deve fornire, infatti, � ottimale solo se il sistema SCADA � in grado di fornire le informazioni corrette sullo stato del processo controllato. Un esempio pratico di quanto descritto pu� essere dato da un sistema SCADA integrato con il sistema gestionale di un magazzino. In questa situazione ipotetica, le decisioni su cosa ordinare e quanto sono prese basandosi sulle informazioni che sono fornite dal sistema SCADA riguardo lo stato del processo controllato. Se queste non fossero corrette, vi sarebbero delle decisioni prese in maniera errata che alla lunga porterebbero ad un danneggiamento economico alla stessa azienda. Questo elencato � solo uno dei molteplici casi in cui � possibile effettuare l'integrazione tra sistema aziendale e quello SCADA, automatizzando la gestione interna. Si possono fare considerazioni simili anche nel caso in cui ci si trova a fornire dei servizi tecnologici tramite web.

Un'altra linea evolutiva scaturita dal processo tecnologico riguarda lo sviluppo di sistema SCADA con larga estensione geografica. In questo caso, un grosso aiuto � stato dato dallo sviluppo delle comunicazioni e le tecnologie legate ad esse. Grazie alle reti ethernet e la tecnologia TCP/IP � stato possibile creare delle infrastrutture di comunicazione sempre pi� affidabili e soprattutto facilmente modificabili secondo le proprie esigenze. Queste, come detto, sono largamente utilizzate nelle strutture di comunicazione di un sistema SCADA. Inoltre si stanno integrando anche funzionalit� legate alla comunicazione radio: sempre pi� spesso si utilizzano comunicazione wireless per le trasmissioni locali, cos� come molti sistemi di controllo utilizzano la rete cellulare sia per comunicare con strutture difficilmente raggiungibili da una rete cablata che per avvisare gli stessi operatori nel caso in cui ci sia bisogno di un intervento urgente.

\subsection{Un nuovo paradigma: la creazione di servizi SCADA}

Una delle ultime evoluzioni scaturite dal miglioramento tecnologico � stato un cambio di approccio alla soluzione del problema della supervisione. Tradizionalmente, come visto, sono realizzati dei sistemi di controllo ad hoc la cui gestione � a carico degli operatori responsabili del processo controllato. Questa soluzione prevede prima una fase di analisi per individuare le esigenze di controllo, definire successivamente i requisiti del sistema e la sua architettura per poi, alla fine, passare alla fase di installazione hardware ed implementazione software. Il tutto chiaramente comporta un lavoro molto lungo e oneroso. 

Negli ultimi tempi, invece, si � applicata un diverso approccio per la ricerca della soluzione al problema sopra citato: il focus � stato spostato sulla realizzazione di servizi SCADA che rispondano a esigenze di tipo funzionale. In questo modo il servizio coinvolge due tipologie di operatori, in maniera diversa. Chi crea il servizio, ossia il fornitore, gestisce e controlla direttamente il sistema SCADA e dovr� occuparsi dei problemi architetturali e progettuali. Il committente, ovverosia il fruitore del servizio, non sar� coinvolto nella realizzazione dell'impianto, ma avr� il compito di gestire i problemi organizzativi nati dall'introduzione del sistema di controllo nella rete aziendale. Una configurazione tipica di un servizio SCADA � mostrata nella figura \ref{fig:scadaservices}, dove � possibile notare la separazione tra il fornitore ed il fruitore del servizio stesso.

\begin{figure}[tbh]

\centerline{\includegraphics[scale=0.75]{Immagini/architettura_servizi_scada.pdf}}
\caption{Architettura di un servizio SCADA}\label{fig:scadaservices}

\end{figure}

Come si pu� notare dallo schema riportato, il sistema di elaborazione viene gestito dal fornitore. Questo sar� messo in comunicazione con le restanti componenti tramite una rete realizzata nelle vicinanze del processo ed accessibile tramite dei canali di comunicazioni simili a quelli utilizzati nei sistemi distribuiti. Il fruitore potr� accedere alle funzionalit� di supervisione e controllo del sistema SCADA tramite le interfacce HMI messe a disposizione dal fornitore.

Ovviamente, occorre tenere anche in considerazione gli aspetti legati all'affidabilit� del servizio, alle prestazione e, soprattutto, alla sicurezza. Se prima ci trovavamo spesso al cospetto di un sistema chiuso, creato ad hoc per rispettare le proprie esigenze, ora l'adozione di un servizio SCADA comporta il condividere molti dati con una terza parte esterna, rappresentata in questo caso dal fornitore. Questi dati, molto spesso, sono cruciali poich� permettono la realizzazione del sistema di controllo. Una loro eventuale perdita o manomissione comporterebbe dei danni sia economici che operativi molto elevati.

\chapter{Analisi dei rischi in un sistema SCADA}

\input{Capitoli/sicurezza_scada.tex}

\chapter{SimSCADA: design e implementazione}

In questo capitolo verranno esposte le motivazioni che hanno portato allo sviluppo di SimSCADA. Inizialmente verranno illustrate le idee che si trovano alla base dello sviluppo del gioco. In seguito verranno mostrate le scelte di design e progettazione prese durante l'implementazione del progetto. Verranno descritte alcune delle dinamiche base presenti nel gioco e come sono state sviluppate. Verr� illustrato quali lezioni sono state inserite nel gioco e come possono essere apprese dal giocatore. Infine verr� mostrato come � stato realizzata la raccolta dati per monitorare e valutare le prestazioni del giocatore.

\section{Scelte di game design}

\subsection{Perch� i sistemi SCADA}

Il gioco � stato sviluppato come parte pratica di una tesi magistrale. Vista le difficolt� che possono nascere durante la creazione di un videogioco, si � cercato principalmente di creare un concept di ci� che potrebbe essere uno strumento di formazione in ambito di sicurezza informatica. Questa � una materia che pu� coinvolgerne molte altre. � stata necessaria, quindi, un'attenta analisi iniziale sia dei serious games a tema di sicurezza informatica gi� presenti nel mercato che di altri lavori precedentemente sviluppati (anche non commercializzati). Da questa analisi � scaturita l'assenza di uno strumento del genere applicabile in un contesto industriale e che andasse a coinvolgere i sistemi SCADA. Si � deciso quindi di rivolgere l'attenzione verso questo mondo che, a causa di una percezione influenzata ancora dalle idee del passato, viene ancora considerato come sicuro e privo di minacce, anche se gli episodi registrati negli ultimi anni descrivono un quadro ben diverso.

Il mondo dei sistemi SCADA � caratterizzato da una elevata variet� di fini applicativi: si va dal sistema di controllo per impianti industriali a quelli per il controllo dei trasporti automatizzati, passando per impianti di produzione energetica (centrali nucleari, gasdotti, ecc.). Piuttosto che focalizzarsi solo su una delle molte applicazioni per sistemi del genere, si � preferito mantenere un'ambientazione pi� generica. Questo � stato fatto principalmente per due motivi:

\begin{itemize}

\item innanzitutto riprodurre solo un determinato campo in maniera fedele avrebbe limitato in qualche modo il bacino d'utenza finale;

\item l'intento principale per cui � stato sviluppato il gioco � quello di cercare di educare il giocatore a prendere delle decisioni mirate ed oculate sia in maniera preventiva che al momento dell'emergenza, cercando di dare maggior importanza alle azioni che l'utente deve attuare rispetto agli effettivi strumenti utilizzati, in quanto differenti da situazione a situazione;

\item per aiutare a mantenere il tipo di ambientazione scelto, si � evitato di inserire rimandi a software e tecnologie specifici utilizzati in ambienti SCADA, in quanto questi si differenziano da situazione a situazione, a seconda del processo da controllare, l'impianto su cui si agisce e tutti gli altri elementi che sono stati elencati in \ref{sec:architecture}.

\end{itemize}

Ai fini di non ridurre in maniera eccessiva l'utenza finale, si sono inserite all'interno del gioco delle nozioni su cosa sia un sistema SCADA ed altre informazioni relativi ad essi, oltre a tutte le lezioni riguardanti i pericoli e gli attacchi informatici a cui un sistema di questi tipo pu� essere sottoposto. Sono poi presenti le lezioni sugli attacchi informatici che sistemi di questo tipo possono subire e quali azioni si possono effettuare per mitigare delle situazioni potenzialmente pericolose.

\subsection{La tipologia di gioco: tycoon}

Per ottemperare all'intento prefissato di dar risalto alle azioni che l'utente deve intraprendere durante la propria partita, si � scelto di sviluppare un gioco di genere gestionale o tycoon, ispirandosi in particolar modo alla serie di titoli di questo genere sviluppati nel corso degli anni, quali ``Roller Coaster Tycoon'' e ``SimCity''. Lo scenario tipico di questi giochi prevede che l'utente si trovi a capo di una determinata realt� economica (un parco di divertimento, una citt� e cos� via, a seconda dell'ambientazione) e deve cercare di gestirla nel migliore dei modi (portandola ad esempio ad aumentare gli incassi o completando altri obiettivi imposti dal sistema), cercando di far fronte con le proprie azioni ai vari problemi che si presentano nel corso della partita.

Da questo punto di vista, si � cercato di riprenderne anche l'impostazione grafico, adottando uno stile retro, di rimando alle versioni dei suddetti titoli sviluppati a cavallo degli anni `80 e `90 (periodo caratterizzato da un massiccio sviluppo di questa tipologia di giochi). Inoltre questo stile permette di mantenere il gioco leggero, graficamente parlando, rendendolo utilizzabile anche su macchine con basse performance. 

\subsection{La piattaforma di destinazione: WebGL}

La piattaforma su cui far eseguire il gioco una volta completato � stata oggetto di diverse analisi. Nella fase iniziale di sviluppo, si era deciso di implementare il gioco per piattaforma Windows, essendo questo il sistema operativo pi� diffuso, soprattutto in ambito di videogames. Durante lo sviluppo del gioco, per�, � stato considerato che, sebbene diffusa, sviluppare il gioco solo per piattaforma Windows sarebbe stato comunque limitante per lo scopo finale del progetto. � stato deciso, quindi, di cambiare la piattaforma di destinazione, affidandosi all'utilizzo delle API WebGL.

Queste interfacce di programmazione, scritte in linguaggio JavaScript, permettono l'esecuzione di applicazioni 2D e 3D tramite web browser, senza richiedere l'utilizzo di alcun tipo di plug-in da parte dell'utente, facendo eseguire il codice relativo alla componente grafica direttamente alla GPU del sistema utente. Sono state rilasciate inizialmente nel 2011 e da allora si sono diffuse largamente, venendo integrate da altri standard web. Recentemente diversi motori grafici hanno iniziato a supportarne lo sviluppo (Unity e Unreal Engine tra i pi� famosi) permettendo ulteriormente il diffondersi di questa tecnologia.

Le ragioni per cui � stato effettuato il cambio della piattaforma sono diverse, quelle che hanno avuto un peso maggiore nella decisione sono state:

\begin{itemize}

\item la tecnologia WebGL pu� essere eseguita su qualsiasi sistema, a prescindere dal sistema operativo installato su di esso. Basta, infatti, l'utilizzo di un web browser che ne supporti l'utilizzo (ovviamente il sistema deve essere sufficientemente potente per garantire il corretto funzionamento del gioco, ma nel caso in questione non ci si � soffermati pi� di tanto su questo aspetto, in quanto il gioco presenta una grafica molto minimale che non richiede un utilizzo eccessivo di risorse da parte del sistema);

\item dovendo effettuare una raccolta dati sulle performance del giocatore, in modo da poter estrapolare successivamente i dati che permettano di capire se l'apprendimento del giocatore sta avvenendo in maniera corretta, utilizzare una tecnologia web, come appunto WebGL, permette una gestione pi� semplice di questo aspetto. Infatti, mentre il codice grafico � eseguito in locale, il codice di controllo del gioco viene eseguito dal server su cui � stato caricato. Per la raccolta dati, quindi, � stato sufficiente creare i file necessari all'interno del server e collezionarli, senza richiedere alcun tipo di azione da parte del giocatore.

\end{itemize}

\subsection{Il motore di gioco: Unity}

La scelta della piattaforma di sviluppo � stata presa senza troppi dubbi, contrariamente agli altri aspetti fin'ora esposti. Subito, infatti, � stato deciso di sviluppare il gioco grazie all'ausilio del software Unity, un motore grafico multipiattaforma, che permette la creazione di applicazioni grafiche sia 2D che 3D. Inizialmente si � tentato di approcciare lo sviluppo del gioco tramite l'utilizzo del framework Phaser.io, ma questa idea � stata rapidamente scartata per diversi motivi tra cui:

\begin{itemize}

\item Phaser.io � un framework di gioco nato di recente e creato unicamente per applicazioni 2D. Unity, d'altro canto, � un motore grafico in circolazione da molto pi� tempo e che permette di sviluppare applicazioni sia in 2D che in 3D. Dato che lo stile grafico del gioco � stato deciso solamente in un secondo momento, avere una piattaforma di sviluppo che non ponesse dei limiti da questo punto di vista � stato decisivo;

\item il fatto che Unity abbia un tempo di vita maggiore, fa si che esistano molti pi� asset grafici, sviluppati da terze parti, liberamente utilizzabili e che hanno permesso di velocizzare lo sviluppo in alcuni momenti;

\item Unity permette di cambiare la piattaforma di destinazione, a prescindere dal codice che si � prodotto fino a quel momento. Questa funzionalit� � stata presa in considerazione nella scelta ed � stata, come si � visto, sfruttata durante lo svolgimento del lavoro;

\item Unity permette di sviluppare codice in diversi linguaggi di programmazione (C, C++, C\#) mentre Phaser.io � scritto solamente per linguaggio JavaScript. Data la maggior familiarit� dello sviluppatore con il linguaggio C\# (utilizzato poi nello sviluppo del codice), Unity � sembrata la scelta pi� adatta per iniziare in maniera pi� veloce lo sviluppo del gioco.

\end{itemize}

\subsection{L'IDE: Visual Studio}

Per quanto riguarda l'IDE di sviluppo, la scelta di Unity come motre grafico ha condizionato in maniera decisiva la scelta di quest'ultimo. Sebbene, infatti, sia possibile utilizzare qualsiasi IDE che permette lo sviluppo di codice C\#, Visual Studio, a differenza degli altri software, possiede diverse integrazioni con il motore in questione che sono di supporto nello sviluppo, rendendo di fatto questa scelta quasi obbligata. 

\section{La raccolta dati}

Prima di iniziare lo sviluppo del gioco, � stata affrontata una fase di raccolta dati. Lo scopo era quello di individuare quale fosse il modo migliore per rappresentare l'ambiente SCADA all'interno di un videogioco. In particolare si voleva evitare di rendere il gioco troppo tecnico e complesso da punto di vista operativo, in quanto avrebbe limitato, come detto, il bacino di utenza finale e sarebbe stato eventualmente di difficile comprensione per il giocatore.

L'idea di gioco iniziale era quella di creare un simulatore di un sistema SCADA. Il giocatore avrebbe interagito con esso tramite l'interfaccia HMI e avrebbe dovuto fronteggiare diversi tipi di attacchi informatici. Per questo motivo sono stati presi in esame una serie di software per interfacce HMI, tra cui WinLog CC e STEP7. Quest'ultimo, in particolare � sviluppato dalla Siemens e fa parte della famiglia di prodotti Simatic, comprendente sia software che hardware per il controllo di sistemi di automazione. Analizzati i software, per� si � capito che emulare un sistema SCADA sarebbe stato complesso, in quanto era necessario dover programmare i diversi PLC che avrebbero controllato il processo (emulato anch'esso tramite computer). 

Scartata quindi l'ipotesi dell'emulatore, si � deciso di porre il focus del gioco sulle diverse minacce che possono coinvolgere un sistema SCADA. Dalle ricerche svolte, infatti, sugli attacchi informatici effettuati ai danni di un sistema di controllo, � emerso che esiste ancora l'errata convinzione di ritenere l'ambiente SCADA sicuro dagli attacchi esterni in quanto tecnologicamente diverso, senza rendersi conto che, in realt�, con il passare degli anni le tecnologie del mondo domestico sono entrate a far parte di quello industriale, rendendoli molto pi� simili di quanto si pensi.

Partendo da ci�, � stata formulata l'idea di un gioco gestionale in cui l'utente si trova a gestire la sicurezza di un ambiente SCADA, fronteggiando diversi tipi di attacchi e cercando di prevenire il fallimento economico per colpa di essi.

\section{La storia nel gioco}

Il gioco si sviluppa all'interno di una societ� fornitrice di servizi SCADA. Il giocatore � stato appena messo a capo del reparto di sicurezza e dovr� gestire i vari aspetti legati ad essa. Quindi non dovr� solo evitare di subire attacchi informatici, ma avr� il compito anche di scovare eventuali infiltrati tra i dipendenti che si troveranno a lavorare all'interno dei locali da proteggere.

Ma mano che riuscir� a difendere i sistemi aziendali, il giocatore guadagner� soldi e aumenter� la propria reputazione. Questo gli permetter� di acquistare migliorie difensive per aiutarlo nel proprio compito. Una volta raggiunto un livello di reputazione e soldi sufficientemente alto, lo scenario si concluder� con la vittoria e si potr� accedere la secondo livello, in cui dovr� fronteggiare minacce pi� pericolose. Se, invece, non dovesse riuscire a proteggere l'azienda dagli attacchi e subisse troppi attacchi, con relative perdite economiche, il gioco terminer� con una sconfitta e il giocatore dovr� ricominciare lo scenario dall'inizio.

La storia � stata mantenuta molto basilare in quanto, tipicamente, nei giochi gestionali non ha una particolare rilevanza. In questo tipo di giochi vi sono, usualmente, degli obiettivi che il giocatore pu� completare nel corso del gioco (raggiungere un determinato livello di introiti economici, effettuare un determinato numero di azioni positive, ecc.), ma in questo caso sono stati estromessi nella programmazione per questioni legate al tempo necessario per idearle ed implementarle.

\section{Struttura del gioco}

Il gioco � ambientato in un'unica mappa, al cui interno si svolgono tutte le azioni del gioco. La mappa � divisa in due locali principali: la stanza server e la stanza degli operatori. Durante lo svolgimento della partita verranno generati automaticamente dei personaggi, raffiguranti i dipendenti dell'azienda. Questi avranno dei lavori da svolgere, la cui durata � indicata da una progress bar presente sopra la testa del personaggio, e il giocatore dovr� impedire che venga interrotto. Se ci riuscir�, ricever� un bonus economico, altrimenti subir� una perdita di soldi.

Non tutti i lavoratori, per�, sono reali: fra di essi, infatti, possono nascondersi dei personaggi ostili, il cui compito sar� quello di effettuare un attacco nei confronti dell'azienda. In questo caso il giocatore dovr� fermarli prima che riescano nel loro intento, altrimenti subir� una perdita economica e, se necessario, dovr� rimettere in funzione il sistema. Per individuare le minacce, il giocatore potr� interagire con i vari personaggi presenti nello scenario e avr� la possibilit� di avvalersi di aiuti acquistabili tramite lo store interno.

Oltre alle minacce fisiche, il giocatore dovr� difendersi anche dalle minacce remote, il cui attacco pu� avvenire in qualunque momento. Per evitarle, sar� compito del giocatore attivare le difese da remoto (firewall, ids, ecc.) e migliorarne il tasso di efficacia, sempre tramite acquisti nello store. 

Il giocatore, oltre che con i personaggi, potr� interagire anche con altre attrezzature (pc, server, ecc.) tramite cui eseguire diverse azioni, sia preventive, ma anche di riparazione, nel caso in cui non sia riuscito a sgominare una minaccia in tempo.

Un'altra delle componenti che sar� d'aiuto durante la partita �, come anticipato, lo store: tramite esso, il giocatore potr� acquistare nuove difese, migliorare quelle che possiede gi� e, inoltre, eseguire campagne di assunzione del personale, in modo da aumentare gli introiti dell'azienda.

\section{Game tutorial}

Prima di iniziare a giocare, viene presentata l'opportunit� di svolgere un tutorial. Questo serve, come in questo tipo di giochi, a presentare le dinamiche base del gioco, mostrare con quali oggetti � possibile interagire, oltre a fornire nozioni sulle minacce da cui dovr� difendersi. 

Il tutorial � parte integrante dell'esperienza di formazione fornita dal gioco, in quanto sono presenti anche delle informazioni sui sistemi SCADA e cosa siano. Questo � stato necessario in quanto non conoscendo in anticipo la formazione del giocatore sull'argomento � importante che i concetti base siano esposti. Questi poi saranno ampliati nel corso della partita, grazie anche alle lezioni inserite nel gioco.

\section{Le lezioni}

Il gioco � suddiviso in due livelli, creati in modo che al passaggio dal primo al secondo la difficolt� aumenti. Durante la partita al giocatore verranno impartite delle lezioni in relazione alle minacce che subir�. Ognuna di esse, infatti, potr� essere effettuata attraverso un diverso tipo di attacco: attacco \mitm, phishing, malware e cos� via, fino ad arrivare anche ad attacchi complessi e/o combinati. Ogni volta che il giocatore subir� un nuovo attacco, verr� subito mostrata la lezione corrispondente in cui saranno anche esposte le misure difensive che dovr� intraprendere.

Essendo diviso in due livelli, nel primo gli attacchi che subir� saranno pi� semplici, mentre nel secondo sono introdotti anche attacchi complessi, a cui il giocatore dovr� reagire con una combinazione di diverse azioni difensive.

Oltre alle lezioni sugli attacchi, sono fornite anche lezioni pi� approfondite su cosa sia un sistema SCADA, sulle componenti che possono trovarsi in un'architettura tipica e sulle difese che � possibile implementare per proteggerlo.

Le lezioni sono accessibili in ogni momento, sia all'interno del gioco (interagendo con il telefono sar� possibile aprire il quaderno al cui interno sono raccolte tutte le lezioni), sia dal men� principale del gioco. In questo modo, se ne dovesse avere bisogno, l'utente potr� rivedere le nozioni di teoria durante la partita, ma anche ripassarle in un secondo momento.

\section{Gestione degli asset di gioco}

Tutti gli asset grafici e audio del gioco sono gestiti automaticamente da Unity, cos� come il loro caricamento. Una volta importati dentro l'editor, infatti, � possibile gestirne il caricamento tramite la classe \cmd{Resources}, in grado di gestire diversi tipi di file multimediali e testuali. Tra gli asset utilizzati sono ad annoverare anche i file di testo al cui interno sono trascritte le lezioni, quelli con i messaggi di sistema, cos� come i file \cmd{.json} utilizzati per creare le varie classi create per il gioco. 

Altri asset caricati sempre tramite la classe \cmd{Resources} sono le immagini utilizzate per raffigurare i personaggi all'interno della mappa e quelle necessarie per raffigurare le loro animazioni di movimento.

\section{Raccolta dati giocatore}

Durante il gioco, sono raccolte delle statistiche sulle azioni del giocatore. Principalmente sono monitorati due parametri: i tempi di reazione nel momento in cui l'utente intraprende delle azioni di risposta all'esecuzione di una minaccia e le azioni di prevenzione eseguite durante il corso di tutta la partita. 

Nel primo caso, lo scopo � di osservare innanzitutto se il giocatore riesce a migliorare i tempi di reazione man mano che la partita procede. Questi tempi sono registrati e illustrati in una tabella, osservabile sia durante la partita che alla fine della stessa. L'andamento dei tempi di risposta, inoltre, pu� fornire un'informazione sulla qualit� dell'apprendimento. Se i fatti i tempi per una determinata tipologia di attacco rimangono invariati nel corso del tempo (o addirittura aumentano) si pu� presupporre che il giocatore abbia difficolt� nel capire in fondo quel tipo di attacco e sapr� quale lezione rivedere per colmare le proprie lacune.

La registrazione delle azioni di risposta, invece, viene effettuata per capire in quale modo il giocatore si comporta nel momento di gioco che intercorre tra una minaccia e l'altra (ossia quei momenti in cui gli attacchi sono in preparazione). Questa registrazione � effettuata in relazione alla ``minaccia in tendenza''. Periodicamente, infatti, sar� scelto un particolare tipo di attacco che sar� di tendenza per un determinato lasso di tempo. Se il giocatore effettuer� delle azioni preventive (acquisti, attivazione o disattivazione di determinati sistemi di difesa) utili a contrastare quel tipo di minaccia � lecito presupporre che abbia recepito la lezione inerente ad essa, altrimenti � probabile che debba rivedere le informazioni presenti all'interno del gioco.

 Un'altra tipologia di azione registrata � quella di ``contrasto'' alle minaccia gi� eseguite. In questa situazione, a seconda dell'attacco subito, verranno analizzate le scelte effettuate dal giocatore per porre rimedio al danno. Anche in questo caso, se le scelte effettuate risultano non in linea con quelle richieste per fermare quel determinato tipo di attacco, � probabile che il giocatore debba rivedere le lezioni inerenti ad esso.
 
 Tutte queste informazioni sono raccolte e scritte nel log del giocatore, creato e salvato all'interno del server. Questo file di testo viene costantemente aggiornato sia con queste informazioni, ma anche con gli altri eventi generati dal gioco.


\chapter{Manuale dello sviluppatore}

In questo capitolo sono inserite tutte le informazioni necessarie per chiunque voglia modificare la struttura del gioco, aggiungendo livelli, minacce, messaggi e quant'altro. Nella prima parte � esposta la modalit� di installazione del gioco, in modo da riuscire ad eseguirlo in locale con fini di sviluppo \ref{sec:howto}. Successivamente, nei paragrafi \ref{sec:noncodecontents}, \ref{sec:classes} e \ref{sec:prefabs} sono fornite informazioni riguardo sia i file di codice che non, utilizzati all'interno del gioco e su cui eventualmente agire per apportare modifiche riguardo la logica del gameplay. In \ref{sec:folderstructure} � mostrata la struttura delle cartelle di gioco, esposta cos� come appare all'interno dell'editor di Unity. Infine, nell'ultima sezione (\ref{sec:addlevel}) � esposto come � possibile aggiungere dei livelli al gioco.

\section{Guida d'installazione}\label{sec:howto}

Il gioco, al momento della stesura di questo report, non � da installare ed � disponibile all'indirizzo \url{http://simscada.sfcoding.com/SIMSCADA/}. Se il sito di hosting non dovesse essere pi� raggiungibile, � possibile eseguirne una versione in un server locale. Questo pu� essere utile anche per effettuare test in caso di modifiche del codice di controllo, aggiunte di livelli e cos� via. Per far s� che il gioco funzioni in locale, per�, sono necessari alcuni passaggi.

\begin{description}

\item[Installazione di un server di gioco] Per eseguire il gioco, � sufficiente avere un web server Apache installato sulla propria macchina e copiare i file prodotti da Unity al momento della build, ossia il contenuto delle cartelle \code{Build} e \code{TemplateData}, pi� il file \code{index.html}, all'interno della cartella \code{htdocs} del server. Esistono diversi metodi per installare un server Apache, a seconda anche del sistema operativo su cui si sta lavorando, ma una soluzione veloce, semplice e soprattutto valida per tutti i casi � quella fornita dal software \code{XAMPP}. 

Questo software, disponibile al link \url{https://www.apachefriends.org/it/download.html}, � disponibile per Windows, MacOS e Linux e permette di installare velocemente tutte le componenti necessarie per eseguire un web server sulla propria macchina in locale. 

Una volta terminata la procedura d'installazione di \code{XAMPP}, sar� necessario individuare la cartella dove sono stati salvati i file. Normalmente, quelle predefinite sono:

\begin{description}

\item[Windows:] C:/xampp

\item[Linux:]  /opt/xampp

\item[MacOS:]  /Applications/XAMPP/xamppfiles

\end{description}

\item[Creazione della cartella di gioco] Trovata la cartella di installazione di \code{XAMPP}, sar� necessario individuare la cartella\code{htdocs}. Al suo interno � consigliato creare un'ulteriore cartella, denominata ad esempio \code{SIMSCADA}. Qui dovranno essere copiati tutti i file prodotti da Unity al momento della build.

\item[Download del progetto] Tutti i file del progetto Unity sono salvati nella repository GitHub raggiungibile all'indirzzo \url{https://github.com/serranda/SecuritySeriousGame}. � possibile ottenerne una copia sia utilizzando il comando \code{git clone https://github.com/serranda/\\SecuritySeriousGame} (per utilizzare questo comando � necessario installare il client Git nel proprio sistema, per questo passaggio si rimanda alle guida ufficiali presenti nel sito di Git, \url{https://git-scm.com/downloads}), ma anche tramite il pulsante download presente nella pagina della repository ( in questo caso verr� effettuato il download di un file .zip. Una volta ottenuta, sar� necessario estrarne il contenuto)

\item[Installazione di Unity] Per installare l'editor di Unity sul proprio sistema, il modo pi� semplice � quello di utilizzare lo Unity Hub creato per aiutare nella gestione delle installazioni e dei progetti creati con l'editor. Questo programma pu� essere ottenuto vistando il forum ufficiale al link \url{https://forum.unity.com/threads/unity-hub-v2-0-0-release.677485/}. Una volta installato Unity Hub, � possibile scegliere la versione dell'editor che si vuole installare. Per questo progetto � consigliato utilizzare le versioni 2018.4.x LTS, in quanto sono quelle che garantiscono la compatibilit� con tutti i file del progetto. Versioni precedenti non sono utilizzabili, mentre per quanto riguarda quelle successive non se ne garantisce il corretto funzionamento con il progetto intero.
Al momento dell'installazione dell'editor � importante selezionare di voler installare anche il modulo aggiuntivo denominato \code{WebGL Build Support}, con cui sar� possibile creare la build effettiva del gioco.
Una volta installato l'editor, � possibile importarvi al suo interno i file del progetto, contenuti all'interno della cartella \code{SIMSCADA}.

\item[Creazione della build] Una volta importati i file del progetto, � possibile aprirlo all'interno dell'editor. Se si vogliono apportare delle modifiche, � possibile farlo. Una volta terminate, � necessario creare la build finale, da caricare successivamente nel server locale creato tramite XAMPP. Per fare ci�, bisogna selezionare la voce \code{Build Settings}, contenuta all'interno del tab \cmd{File}. A questo punto si aprir� una finestra in cui sar� necessario impostare quali scene si vogliono includere nella build finale e per quale piattaforma effettuarla. Occorrer� quindi selezionare la piattaforma WebGL dal men� posto sulla sinistra della finestra e premere sul tasto \code{Build}. A questo punto, dopo aver scelto la cartella all'interno di cui verranno generati tutti i file, la compilazione inizier� e sar� necessario attenderne il termine.

\item[Copia della build nel server] Terminata la compilazione, � possibile prenderne il contenuto e spostarlo all'interno della cartella \code{SIMSCADA}, creata precedentemente nella directory \code{htdocs} di \code{XAMPP}. � necessario, inoltre, spostare anche la cartella \code{PHP} (situata all'interno della directory del progetto ottenuta tramite GitHub), contenente i file degli script PHP necessari per eseguire operazioni sul server. Una volta eseguite queste azioni � possibile utilizzare un qualsiasi web browser che supporti l'utilizzo di WebGL (per controllare che il proprio browser supporti l'esecuzione di applicazioni WebGL � possibile collegarsi al sito \url{https://get.webgl.org/} ed effettuare il test ) e collegarsi all'indirizzo \url{localhost/SIMSCADA/}. Il gioco partir� non appena terminato il caricamento della pagina.

\end{description}

\section{Contenuti del gioco senza codice}\label{sec:noncodecontents}

Diverse risorse del gioco sono definite attraverso file di testo e json in modo da agevolare le operazioni di gestione durante la partita e/o eventuali loro modifiche durante lo sviluppo. Questi file sono contenuti all'interno della cartella \code{Resources} della directory del progetto. Questo si � reso necessario in modo da poter sfruttare i metodi della classe \code{Resources} e caricare velocemente il contenuto dei file all'interno del gioco. Di seguito saranno descritte le strutture di queste risorse pi� nel dettaglio. 

Di queste risorse sono stati creati anche dei template a cui far riferimento, nel caso in cui se ne voglia aggiungere di nuovi. Questi sono situati sempre nella directory del progetto Unity.

\subsection{Messages}

All'interno del gioco sono presenti diversi messaggi di sistema rivolti all'utente, in modo da informarlo sugli avvenimenti della partita, se sta commettendo qualche azione errata o, ancora, per fornire dei suggerimenti. Vi sono poi i messaggi del tutorial, tramite cui si danno al giocatore le informazioni necessarie per apprendere le dinamiche basilari del gameplay. Ognuno di questi messaggi � creato partendo dal contenuto di due file, un file json ed un file txt.

Nel file json sono contenuti i valori delle propriet� della classe con cui si definiscono i messaggi all'interno del codice (un esempio di file json di questo genere � mostrao nella figura \ref{fig:dialogboxmessagejson}). La classe � stata denominata \code{DialogBoxMessage} ed il suo contenuto � riportato nella figura \ref{fig:dialogboxmessageclass}.

Come � possibile vedere, un oggetto della classe \code{DialogBoxMessage} � composto da cinque propriet�. Le prime quattro 4 sono fornite dal file json e sono:

\begin{itemize}

\item \code{head}, ossia il testo riportato nel titolo del messaggio;

\item \code{bodyPath}, ossia il percorso dove � salvato il testo del messaggio;

\item \code{backBtn}, ossia il testo che viene riportato in uno dei due pulsanti con cui il giocatore pu� interagire (usualmente con questo valore si vuole indicare il pulsante che permette di tornare indietro o annullare un'eventuale azione proposta dal sistema);

\item \code{nextBtn}, ossia il testo che viene riportato nell'altro pulsante (con questo valore, al contrario, si vuole indicare il pulsante che permette procedere con l'azione proposta dal gioco).

\end{itemize}

Queste propriet� sono assegnate durante la creazione dell'oggetto, resa possibile grazie al metodo \code{FromJson} della classe \code{JsonUtility}, messa a disposizione dalle API di Unity (\ref{fig:jsonutility}).

La quinta propriet�, \code{body}, come si pu� vedere la codice della classe riportato in figura \ref{fig:dialogboxmessageclass}, viene creata automaticamente a partire dalla propriet� \code{bodyPath}. Come anticipato, \code{bodyPath} contiene il valore del percorso in cui � salvato il file txt in cui � riportato il testo completo del messaggio. � stato deciso, infatti, di non inserire l'intero contenuto del messaggio nel file json, ma di salvarlo in un file a parte, tenendo nota per� della sua posizione (relativa al working tree del progetto di Unity).

Questa strategia, applicata anche per i file delle \code{Lesson} (come si vedr� successivamente in \ref{sec:lessons}) � stata ideata per evitare di incappare in errori durante la fase di deserializzazione del file json. In questo modo, inoltre, � possibile modificare il contenuto del messaggio con un qualsiasi editor di testo (� importante che la codifica del file txt sia impostata in \code{UTF-8}, altrimenti l'editor di Unity non riconsocer� il contenuto e dar� un errore al momento della creazione dell'oggetto) senza dover cambiare la struttura del file json.

\begin{figure}[tb]
\HRule
\begin{lstlisting}
{
	"head": <MessageHead>,
	"bodyPath": <MessageBodyPath>,
	"backBtn": <MessageBackBtn>,
	"nextBtn": <MessageNextBtn>
}
\end{lstlisting}
\HRule
\caption{Contenuto del file json da cui sono estratte le propriet� della classe \code{DialogBoxMessage}.\label{fig:dialogboxmessagejson}}
\end{figure}

\begin{figure}[tb]
\HRule
\begin{lstlisting}
using UnityEngine;

public class DialogBoxMessage
{
    public string head;
    public string bodyPath;
    public string backBtn;
    public string nextBtn;

    public string body =>
        string.IsNullOrEmpty(bodyPath) 
            ? string.Empty 
            : Resources.Load<TextAsset>(bodyPath).text;
}
\end{lstlisting}
\HRule
\caption{Contenuto della classe \code{DialogBoxMessage}.\label{fig:dialogboxmessageclass}}
\end{figure}

\begin{figure}[tb]
\HRule
\begin{lstlisting}
DialogBoxMessage dialogBoxMessage = JsonUtility.FromJson<DialogBoxMessage>(jsonFile.text);
\end{lstlisting}
\HRule
\caption{Codice tramite cui sono creati gli oggetti della classe \code{DialogBoxMessage}.\label{fig:jsonutility}}
\end{figure}

\subsection{Lesson}\label{sec:lessons}

Anche per le lezioni � stato utilizzato un approccio simile a quello visto per i \code{DialogBoxMessage}. In questo caso le propriet� definite per la classe sono solamente tre (\ref{fig:lessonclass}), di cui quelle riportate nel file json sono \code{id} e \code{textPath} (\ref{fig:lessonjson}).

Come nel caso dei \code{DialogBoxMessage}, anche il testo delle \code{Lesson} � salvato in un file di testo a parte, la cui posizione � memorizzata con la propriet� \code{textPath}. Questa verr� poi utilizzata per generare, al momento della creazione dell'oggetto, il valore della propriet� \code{textBody}. 

L'altra propriet�, \code{id}, � utilizzata per identificare l'argomento associato alla lezione in questione. Il suo valore, inoltre, � impiegato per rinominare i pulsanti tramite cui � possibile selezionare la lezione a cui si vuole accedere (\ref{fig:notebookImage})

\begin{figure}[tb]
\HRule
\begin{lstlisting}
{
  "id":"LessonID",
  "textPath":"LessonBodyPath"
}
\end{lstlisting}
\HRule
\caption{Contenuto del file json da cui sono estratte le propriet� della classe \code{Lesson}.\label{fig:lessonjson}}
\end{figure}

\begin{figure}[tb]
\HRule
\begin{lstlisting}
using UnityEngine;

public class Lesson
{
    public string id;
    public string textPath;

    public string textBody =>
        string.IsNullOrEmpty(textPath)
            ? string.Empty
            : Resources.Load<TextAsset>("LessonsBody/" + textPath + "_IT").text;
}
\end{lstlisting}
\HRule
\caption{Contenuto della classe \code{Lesson}.\label{fig:lessonclass}}
\end{figure}

\begin{figure}
\centerline{\includegraphics[scale=0.4]{Immagini/notebookImage}}
\caption{ Screenshot del quaderno presente nel gioco, tramite cui � possibile accedere alle \code{Lesson}. Sulla destra sono visibili i pulsanti, raffigurati come dei segnalibri, che permettono la visualizzazione del testo inerente alla \code{Lesson} desiderata. }\label{fig:notebookImage}
\end{figure}

\subsection{ItemStore}

\section{Classi}\label{sec:classes}

\section{Prefab}\label{sec:prefabs}

\section{Struttura delle cartelle}\label{sec:folderstructure}

\section{Come aggiungere un livello}\label{sec:addlevel}

\chapter{Risultati}

\chapter{Conclusioni}

% bibliografia scritta "a mano"
%\input{biblio.tex}

% se la bibliografia � stata scritta (usando Bibtex) nel file biblio.bib allora commentare la riga precedente e scommentare le due righe seguenti
%\bibliographystyle{torsec}
%\bibliography{biblio}

\printbibliography

\end{document}
