\begin{table}[tbh]
\begin{tabularx}{\textwidth}[c]{@{}XX@{}}
\toprule
\textbf{PARADIGMI PRE-STUXNET} & \textbf{PARADIGMI POST-STUXNET} \\ \midrule
I sistemi possono essere isolati efficacemente dalle reti esterne, eliminando il rischi di incidenti informatici & I sistemi di controllo sono ancora soggetti alla natura umana: un forte perimetro difensivo pu� sempre venire oltrepassato a causa della superficialit� nel trattare la sicurezza \\ \midrule
I PLC che non eseguono sistemi operativi moderni non sono vulnerabili agli attacchi informatici & I PLC possono essere degli obiettivi concreti per i malware \\ \midrule
I macchinari utilizzati per operazioni speciali beneficiano del paradigma "security through obscurity". Dato che la documentazione degli ICS non � disponibile, � impossibile creare un attacco contro di loro & Le risorse per creare un attacco di successo ed altamente specializzato contro un ICS sono tutte a disposizione degli hacker \\ \midrule
Per proteggere la rete di un ICS da un attacco � sufficiente l'utilizzo di firewall e IDS & L'utilizzo di molteplici vulnerabilit� allo zero-day per effettuare un attacco indicano che i metodi difensivi basati su blacklist non sono pi� sufficienti ed occorre considerare tecnologie basate su whitelist per difendersi contro delle vulnerabilit� ancora sconsociute \\ \bottomrule
\end{tabularx}
\caption{Differenze dei paradigmi difensivi SCADA prima e dopo l'avvento di Stuxnet (fonte: \cite{Knapp_2015_7})}
\label{tab:stuxnet_differences}
\end{table}