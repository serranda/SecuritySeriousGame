\documentclass[pdfa,cucitura] {toptesi}

\usepackage[latin1]{inputenc}

% !TEX encoding = IsoLatin

% per inserire uno spazio "fantasma" nella definizione di un'abbreviazione
\usepackage{xspace}

% per inserire un DOI senza problemi coi caratteri "strani" ivi presenti
\usepackage{doi}
\renewcommand{\doitext}{DOI }% originally was "doi:"

% per inserire correttamente le unit� di misura SI (incluse quelle binarie)
\usepackage[binary-units]{siunitx}
% se si desidera usare / invece che la potenza -1 per indicare "al secondo"
\sisetup{per-mode=symbol}

% per inserire codice di programmazione complesso
\usepackage{listings}% per inserire codice di programmazione complesso
\lstset{
basicstyle=\ttfamily,
columns=fullflexible,
xleftmargin=3ex,
breaklines,
breakatwhitespace,
escapechar=`
}

% modify some page parameters
\setlength{\parskip}{\medskipamount}

% riga orizzontale
\newcommand{\HRule}{\rule{\linewidth}{0.2mm}}
% esempio di creazione di semplici abbreviazioni
\newcommand{\ltx}{\LaTeX\xspace}
\newcommand{\txw}{TeXworks\xspace}
\newcommand{\mik}{MikTex\xspace}
\newcommand{\html}{HTML\xspace}
\newcommand{\xhtml}{XHTML\xspace}

% esempio di creazione di un'abbreviazione con un parametro (il cui uso � indicato da #1)
\newcommand{\cmd}[1]{\texttt{#1}\xspace}
% per citare un RFC, es. \rfc{822}
\newcommand{\rfc}[1]{RFC-#1\xspace}
% per citare un file (es. \file{autoexec.bat}) o una URI fittizia (es. \file{http://www.lioy.it/})
% per le URI vere usare \url o \href
\newcommand{\file}[1]{\texttt{#1}\xspace}
% per inserire codice di esempio in-line
\newcommand{\code}[1]{\lstinline|#1|}
% importante per i pathname Windows perch� non si pu� usare \ essendo un carattere riservato di Latex
\newcommand{\bs}{\textbackslash}
% definizione di un termine: formattazione ed inserimento nell'indice
\newcommand{\tdef}[1]{\textit{#1}\index{#1}}
% meta-termine, usato tipicamente nelle definizioni dei tag
\newcommand{\meta}[1]{\textit{#1}}
% abbreviazioni in inglese
\newcommand{\ie}{i.e.\xspace}
\newcommand{\eg}{e.g.\xspace}

\newcommand{\tabitem}{~~\llap{\textbullet}~~}

%comando per [d0x3d!]
\newcommand{\doxed}{[d0x3d!]\xspace}

%comando per i nomi degli attacchi
\newcommand{\mitm}{Man in the Middle\xspace}
\newcommand{\mitma}{MitM\xspace}
\newcommand{\dos}{Denial of Service\xspace}
\newcommand{\dosa}{Dos\xspace}
\newcommand{\ddos}{Distributed Denial of Service\xspace}
\newcommand{\ddosa}{DDos\xspace}

\begin{document}

Il seguente documento � un riassunto della tesi di laurea magistrale svolta dallo studente Andrea Marchetti, matricola 225277. I relatori della tesi sono il prof. Antonio Lioy e l'ing. Andrea Atzeni.

Scopo del lavoro � stato quello di sviluppare un serious game in ambito di sicurezza informatica. In particolare si � affrontato il tema delle minacce informatiche negli ambienti SCADA. Il risultato del lavoro svolto � stata la creazione di SimSCADA, un security serious game di tipo gestionale per piattaforma WebGL.

Nel primo capitolo � stata eseguita un'analisi delle tematiche della sicurezza informatica e dei sistemi SCADA e di come questi siano diventati nel tempo materie di alto rilievo, come conseguenza dello sviluppo tecnologico.

Nel secondo capitolo viene effettuata un'analisi pi� approfondita dei serious games, in particolare dei security serious games gi� presenti sul mercato, incentrandosi su quelli ritenuti pi� interessanti e peculiari per le loro dinamiche di insegnamenti e di gioco. Questi titoli sono CyberGENIE, \doxed e Anti Phishing Phil.

Successivamente, nel terzo e nel quarto capitolo viene effettuata un'analisi sugli ambienti e le tecnologie SCADA (cosa sono, da quali elementi sono costituiti, quali differenze ci sono con altri sistemi simili) e i rischi che pu� correre in un sistema di questo tipo. A tal proposito � stato effettuato un paragone con gli ambienti domestici, focalizzandosi sulle differenze che vi sono tra la sicurezza domestica e quella industriale e quali sono i punti di vulnerabilit� che possono essere sfruttati. Viene poi proposto un elenco dei possibili attacchi effettuabili ai danni di un sistema SCADA.

Nel quinto capitolo sono esposte le motivazioni che hanno portato alle scelte tecniche su cui si � basato lo sviluppo di SimSCADA. Le motivazioni sono state prese sulla base di studi applicati ai serious games presi in analisi inizialmente e i risultati hanno permesso di definire quali elementi utilizzare per la creazione di SimSCADA. 

Il sesto capitolo raccoglie tutte le informazioni necessarie per il manuale del programmatore, esponendo le parti di codice pi� rilevanti e importanti del gioco. Inoltre sono offerti dei punti guida per aiutare chiunque abbia intenzione di proseguire lo sviluppo del gioco in futuro.

Nel settimo capitolo sono esposti i risultati della raccolta dati effettuata per verificare l'utilizzabilit� del titolo e la sua efficacia didattica. I dati sono stati raccolti facendo provare SimSCADA ad un campione di diciotto studenti universitari con et� compresa tra i 23 e i 26 anni. I risultati mostrano che dopo l'utilizzo del serious game vi � stato un incremento nelle conoscenze dei giocatori, ma ovviamente per essere convalidati definitivamente bisognerebbe condurre una sperimentazione su pi� larga scala.

L'ottavo ed ultimo capitolo contiene le conclusioni, analizzando lo stato attuale dei serious games. Inoltre � stata aggiunta una serie di idee per dei possibili futuri sviluppi del gioco.

\end{document}
