\begin{description}

\item[Installazione di un server di gioco] Per eseguire il gioco, � sufficiente avere un web server Apache installato sulla propria macchina e copiare i file prodotti da Unity al momento della build, ossia il contenuto delle cartelle \code{Build} e \code{TemplateData}, pi� il file \code{index.html}, all'interno della cartella \code{htdocs} del server. Esistono diversi metodi per installare un server Apache, a seconda anche del sistema operativo su cui si sta lavorando, ma una soluzione veloce, semplice e soprattutto valida per tutti i casi � quella fornita dal software \code{XAMPP}. 

Questo software, disponibile al link \url{https://www.apachefriends.org/it/download.html}, � disponibile per Windows, MacOS e Linux e permette di installare velocemente tutte le componenti necessarie per eseguire un web server sulla propria macchina in locale. 

Una volta terminata la procedura d'installazione di \code{XAMPP}, sar� necessario individuare la cartella dove sono stati salvati i file. Normalmente, quelle predefinite sono:

\begin{description}

\item[Windows:] C:/xampp

\item[Linux:]  /opt/xampp

\item[MacOS:]  /Applications/XAMPP/xamppfiles

\end{description}

\item[Creazione della cartella di gioco] Trovata la cartella di installazione di \code{XAMPP}, sar� necessario individuare la cartella\code{htdocs}. Al suo interno � consigliato creare un'ulteriore cartella, denominata ad esempio \code{SIMSCADA}. Qui dovranno essere copiati tutti i file prodotti da Unity al momento della build.

\item[Download del progetto] Tutti i file del progetto Unity sono salvati nella repository GitHub raggiungibile all'indirzzo \url{https://github.com/serranda/SecuritySeriousGame}. � possibile ottenerne una copia sia utilizzando il comando \code{git clone https://github.com/serranda/} \code{SecuritySeriousGame} (per utilizzare questo comando � necessario installare il client Git nel proprio sistema, per questo passaggio si rimanda alle guida ufficiali presenti nel sito di Git, \url{https://git-scm.com/downloads}), ma anche tramite il pulsante download presente nella pagina della repository ( in questo caso verr� effettuato il download di un file .zip. Una volta ottenuta, sar� necessario estrarne il contenuto)

\item[Installazione di Unity] Per installare l'editor di Unity sul proprio sistema, il modo pi� semplice � quello di utilizzare lo Unity Hub creato per aiutare nella gestione delle installazioni e dei progetti creati con l'editor. Questo programma pu� essere ottenuto vistando il forum ufficiale al link \url{https://forum.unity.com/threads/unity-hub-v2-0-0-release.677485/}. Una volta installato Unity Hub, � possibile scegliere la versione dell'editor che si vuole installare. Per questo progetto � consigliato utilizzare le versioni 2018.4.x LTS, in quanto sono quelle che garantiscono la compatibilit� con tutti i file del progetto. Versioni precedenti non sono utilizzabili, mentre per quanto riguarda quelle successive non se ne garantisce il corretto funzionamento con il progetto intero.
Al momento dell'installazione dell'editor � importante selezionare di voler installare anche il modulo aggiuntivo denominato \code{WebGL Build Support}, con cui sar� possibile creare la build effettiva del gioco.
Una volta installato l'editor, � possibile importarvi al suo interno i file del progetto, contenuti all'interno della cartella \code{SIMSCADA}.

\item[Creazione della build] Una volta importati i file del progetto, � possibile aprirlo all'interno dell'editor. Se si vogliono apportare delle modifiche, � possibile farlo. Una volta terminate, � necessario creare la build finale, da caricare successivamente nel server locale creato tramite XAMPP. Per fare ci�, bisogna selezionare la voce \code{Build Settings}, contenuta all'interno del tab \cmd{File}. A questo punto si aprir� una finestra in cui sar� necessario impostare quali scene si vogliono includere nella build finale e per quale piattaforma effettuarla. Occorrer� quindi selezionare la piattaforma WebGL dal men� posto sulla sinistra della finestra e premere sul tasto \code{Build}. A questo punto, dopo aver scelto la cartella all'interno di cui verranno generati tutti i file, la compilazione inizier� e sar� necessario attenderne il termine.

\item[Copia della build nel server] Terminata la compilazione, � possibile prenderne il contenuto e spostarlo all'interno della cartella \code{SIMSCADA}, creata precedentemente nella directory \code{htdocs} di \code{XAMPP}. � necessario, inoltre, spostare anche la cartella \code{PHP} (situata all'interno della directory del progetto ottenuta tramite GitHub), contenente i file degli script PHP necessari per eseguire operazioni sul server. Una volta eseguite queste azioni � possibile utilizzare un qualsiasi web browser che supporti l'utilizzo di WebGL (per controllare che il proprio browser supporti l'esecuzione di applicazioni WebGL � possibile collegarsi al sito \url{https://get.webgl.org/} ed effettuare il test ) e collegarsi all'indirizzo \url{localhost/SIMSCADA/}. Il gioco partir� non appena terminato il caricamento della pagina.

\end{description}