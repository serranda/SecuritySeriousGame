In questo capitolo sono inserite tutte le informazioni necessarie per chiunque voglia modificare la struttura del gioco, aggiungendo livelli, minacce, messaggi e quant'altro. Nella prima parte � esposta la modalit� di installazione del gioco, in modo da riuscire ad eseguirlo in locale con fini di sviluppo \ref{sec:howto}. Successivamente, nei paragrafi \ref{sec:noncodecontents}, \ref{sec:classes} e \ref{sec:prefabs} sono fornite informazioni riguardo sia i file di codice che non, utilizzati all'interno del gioco e su cui eventualmente agire per apportare modifiche riguardo la logica del gameplay. In \ref{sec:folderstructure} � mostrata la struttura delle cartelle di gioco, esposta cos� come appare all'interno dell'editor di Unity. Infine, nell'ultima sezione (\ref{sec:addlevel}) � esposto come � possibile aggiungere dei livelli al gioco.

\section{Guida d'installazione}\label{sec:howto}

Il gioco, al momento della stesura di questo report, � disponibile all'indirizzo \url{http://simscada.sfcoding.com/SIMSCADA/}. Se non dovesse essere pi� disponibile, � possibile eseguirne una versione in un server locale. Questo pu� essere utile anche per effettuare test in caso di modifiche del codice di controllo, aggiunte di livelli e cos� via. Per far s� che il gioco funzioni in locale, per�, sono necessari alcuni passaggi.

\begin{description}

\item[Installazione di un server di gioco] Per eseguire il gioco, � sufficiente avere un web server Apache installato sulla propria macchina e copiare i file prodotti da Unity al momento della build, ossia il contenuto delle cartelle \cmd{Build} e \cmd{TemplateData}, pi� il file \cmd{index.html}, all'interno della cartella \cmd{htdocs} del server. Esistono diversi metodi per installare un server Apache, a seconda anche del sistema operativo su cui si sta lavorando, ma una soluzione veloce, semplice e soprattutto valida per tutti i casi � quella fornita dal software \cmd{XAMPP}. 

Questo software, disponibile al link \url{https://www.apachefriends.org/it/download.html}, � disponibile per Windows, MacOS e Linux e permette di installare velocemente tutte le componenti necessarie per eseguire un web server sulla propria macchina in locale. 

Una volta terminata la procedura d'installazione di \cmd{XAMPP}, sar� necessario individuare la cartella dove sono stati salvati i file. Normalmente, quelle predefinite sono:

\begin{description}

\item[Windows:] C:/xampp

\item[Linux:]  /opt/xampp

\item[MacOS:]  /Applications/XAMPP/xamppfiles

\end{description}

\item[Creazione della cartella di gioco] Trovata la cartella di installazione di \cmd{XAMPP}, sar� necessario individuare la cartella\cmd{htdocs}. Al suo interno � consigliato creare un'ulteriore cartella, denominata ad esempio \cmd{SIMSCADA}. Qui dovranno essere copiati tutti i file prodotti da Unity al momento della build.

\item[Download del progetto] � possibile ottenere

\item[Installazione di Unity]

\item[Creazione della build]

\item[Copia della build nel server]

\end{description}

\section{Contenuti del gioco senza codice}\label{sec:noncodecontents}

\section{Classi}\label{sec:classes}

\section{Prefab}\label{sec:prefabs}

\section{Struttura delle cartelle}\label{sec:folderstructure}

\section{Come aggiungere un livello}\label{sec:addlevel}