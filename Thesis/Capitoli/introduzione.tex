\section{La Sicurezza Informatica}

Nel corso degli ultimi anni l'utilizzo di sistemi informatici ha avuto un utilizzo sempre pi� incrementale, sia in ambito industriale che per quanto riguarda i vari aspetti della vita quotidiana. L'Internet of Things � una realt� che si sta pian piano affermando in ogni aspetto della nostra vita, delineando quindi una sempre pi� grande e complessa rete di sistemi informatici interconnessi tra di loro. Parallelamente a tutto ci� si � reso necessario la definizione e l'implementazione di tecniche e protocolli per garantire la sicurezza informatica. Con il termine "sicurezza informatica" andiamo a identificare l'insieme dei prodotti, dei servizi, delle regole organizzative e dei comportamenti individuali che servono a proteggere i sistemi informatici a fronte di attacchi, danneggiamenti e/o accessi indesiderati, di natura accidentale o meno. Il fine � quello di riuscire a mantenere la confidenzialit�, l'integrit� e la disponibilit� dei dati presenti in un sistema informatico \cite{Rouse_2018}.

L'aumentare dell'importanza della tecnologia nella nostra vita � testimoniato anche da altri dati: nell'annuale report di mercato di Cybersecurity Ventures \cite{Morgan_2019}, viene evidenziato come questo mercato continui ad essere in espansione, andando ad assumere attualmente un valore di circa 120 miliardi di dollari. Molte aziende continuano ad investire nei loro reparti di ricerca e sviluppo contro le minacce informatiche, facendo registrare un generale aumento della quantit� di soldi esborsa. Questo dato va chiaramente preso in considerazione anche rispetto ai danni economici che possono scaturire a seguito di cyberattacchi, i quali possono generare cifre molto pi� alte di quelle investite. � ovvio quindi come allo stato attuale si ricorra sempre di pi� alla tecnologia, cercando di proteggersi nella maniera migliore possibile. 

Purtroppo per� non molto spesso si riesce in questo intento. Nell'ultimo anno si sono registrati numeri che stanno a testimoniare come occorra continuare ad investire risorse in questo ramo. Sono avvenute pi� di 53000 incidenti e oltre 2000 violazioni di dati confermate. Le vittime principali sono state le piccole imprese, molto spesso attaccate per motivazioni economiche, ma non mancano anche quelle di tipo strategico e di spionaggio. Rispetto al 2016, inoltre, si � registrato un lieve aumento tra gli organizzatori degli attacchi di figure operanti che lavoravano all'interno delle stesse aziende attaccate, a testimoniare ulteriormente come le minacce possano avere origine da qualsiasi parte, non necessariamente solo dall'esterno dell'ambiente di lavoro. La maggior parte degli attacchi � stata perpetrata attraverso l'utilizzo di ransomware, dei malware che rendono impossibile l'accesso ai dati alle vittime, a meno che non si effettui un pagamento per permettere lo sblocco delle risorse in questione. Sono stati numerosi anche gli attacchi attraverso le cosiddette tecniche di "ingegneria sociale": pi� di 1400 incidenti con quasi 400 casi di dati persi confermati \cite{Widup_2018}.\pagebreak 

\section{Sistemi SCADA}

La diffusione delle tecnologie IT � andata ad influenzare anche gli ambienti industriali definiti "isolati". Essi hanno basato la maggior parte dei loro protocolli di sicurezza proprio sul fatto che fossero degli ambienti con stretti e controllati rapporti con l'esterno. Questo ormai � un paradigma che si sta via via estinguendo, lasciando spazio ad ambienti industriali pienamente influenzati e coinvolti nell'Internet of Things.

Questo aspetto ha riguardato anche i sistemi SCADA (Supervisory Control and Data Acquisition), ossia quegli impianti di supervisione e controllo utilizzati soprattutto in ambiti industriali (tipicamente di generazione e distribuzione di energia elettrica e acquedotti), ma anche in aeroporti e ferrovie. Questi sistemi sono tipicamente composti da un gran numero di sensori collegati tra loro attraverso un sistema di comunicazione. Il nodo finale � costituito da uno o pi� computer supervisori ed hanno il compito di analizzare i dati raccolti dai sensori. Questi nodi inoltre rappresentano i punti di utilizzo, tramite cui � possibile interfacciarsi con l'impianto SCADA, ma possono non essere gli unici presenti nel sistema.Negli ultimi anni, infatti, si sono cominciate a sviluppare applicazioni per dispositivi handheld, ossia smartphone e tablet, per effettuare operazioni di controllo attraverso di essi. Questi diventano particolarmente utili quando si ha a che fare con sistemi SCADA con una vasta estensione geografica.

Sebbene tutti questi aspetti positivi appena analizzati, in realt� ci sono pi� insidie di quanto si pensi. I protocolli di sicurezza, cos� come quelli di comunicazione, utilizzati attualmente in ambiente SCADA, non tenevano per nulla conto delle minacce dovute al mondo esterno, dato che erano concepiti per ambienti isolati e affidabili, e sono risultati inefficaci per fronteggiare minacce IT. Inoltre le tecnologie introdotte hanno portato con loro anche le varie vulnerabilit� di cui gi� soffrivano, andando quindi ad aumentare i pericoli a cui un sistema SCADA pu� essere sottoposto. A seguito anche di alcuni attacchi avvenuti a discapito di questi sistemi (uno su tutti il caso risalente al 2010 di Stuxnet, virus informatico con cui i governi di USA e Israele infettarono i sistemi di controllo di centrali nucleari iraniane \cite{Zetter_2015} e che recentemente ha colpito anche delle centrali russe \cite{Shamah_2013}), si � reso necessario sviluppare nuovi protocolli, in modo da garantire la sicurezza degli impianti SCADA e pi� in generale quella dei sistemi di controllo industriali.

\section{Obiettivi del lavoro}

I protocolli di sicurezza, e pi� in generale tutti gli strumenti necessari per avere un ambiente informatico sicuro, devono essere utilizzati nella maniera corretta, altrimenti si rischia di ottenere un effetto opposto a quello per cui sono stati destinati. � necessario quindi, oltre ai vari investimenti economici per la ricerca e lo sviluppo, effettuare un'adeguata formazione delle persone, in modo da educarli a come comportarsi in caso di situazioni a rischio.

Questo discorso � applicabile sia in ambito privato che negli ambienti industriali. In questi ultimi casi riuscire a garantire il corretto funzionamento dei sistemi di sicurezza significa riuscire a garantire l'incolumit� di molta gente, sia fisica (pensiamo ai sistemi di controllo di acquedotti, impianti nucleari, ecc.) che quella dei dati a loro associati (se invece pensiamo a sistemi bancari, catastali, ecc.). Purtroppo l'evoluzione tecnologica � serrata e rimanere indietro costituisce un grosso vantaggio per chi commette cybercimini.

Lo scopo della tesi � quello di analizzare una possibile metodologia di formazione, ossia quella dei Serious Games, ed applicarla al campo della sicurezza informatica. In particolare gli argomenti di sicurezza che si affronteranno saranno strettamente legati all'ambiente dei sistemi SCADA. L'elaborato si divide in due parti principali: nella prima parte si effettua un'analisi teorica sul modello di apprendimento basato sull'utilizzo dei Serious Games insieme ad un'analisi di soluzioni gi� esistenti, cos� da vedere come questo metodo possa essere applicato alla materia della sicurezza informatica,. Vengono presentati inoltre i sistemi SCADA, la loro architettura, i loro protocolli di comunicazione, le criticit� che li affligge ed i protocolli di sicurezza esistenti per difenderli. Nella seconda parte ci si concentra nella realizzazione di un Serious Game per simulare un sistema SCADA sotto attacco, costringendo l'utente a valutare attentamente la situazione e prendere le decisioni pi� adatte per contrastare l'emergenza.